\documentclass[a4paper, 11pt]{article}

% --- PACHETTI STANDARD ---
\usepackage[utf8]{inputenc}
\usepackage[T1]{fontenc}
\usepackage{amsmath}
\usepackage{amssymb}
\usepackage{graphicx}
\usepackage[italian]{babel}
\usepackage{fancyvrb}
\usepackage{tabularx} % Per le tabelle
\usepackage{tikz} % Per i diagrammi
\usepackage{algorithm}
\usepackage{algpseudocode}
\usepackage{forest}
\usetikzlibrary{positioning,shapes.geometric}
\usepackage[utf8]{inputenc}
\usepackage[T1]{fontenc}
\usepackage{tikz}
\usetikzlibrary{automata, positioning, arrows, trees, chains, decorations.pathreplacing, calc}
\usepackage[hidelinks]{hyperref}
\usetikzlibrary{shapes, arrows.meta, positioning, calc}

% --- PACCHETTO PER IL CODICE (Mancava questo o la sua configurazione) ---
\usepackage{listings}
\usepackage{xcolor}

% --- CONFIGURAZIONE GRAFICA DEL CODICE ---
\lstset{
    basicstyle=\ttfamily\small,       % Stile del font base
    keywordstyle=\color{blue}\bfseries, % Parole chiave in blu
    stringstyle=\color{red},          % Stringhe in rosso
    commentstyle=\color{gray}\itshape,% Commenti in grigio
    showspaces=false,                 % <--- IMPORTANTE: Nasconde i simboli degli spazi nel codice
    showstringspaces=false,           % <--- IMPORTANTE: Nasconde gli spazi dentro le stringhe "..."
    frame=single,                     % Riquadro attorno al codice
    breaklines=true,                  % A capo automatico
    tabsize=2,                        % Dimensione tabulazione
    language=C,                       % Linguaggio (o C++ o Java)
    morekeywords={label, string},     % Aggiunge parole chiave personalizzate (se servono)
    literate={à}{{\`a}}1 {è}{{\`e}}1 {é}{{\'e}}1 {ì}{{\`i}}1 {ò}{{\`o}}1 {ù}{{\`u}}1 % Accenti
}

% --- IMPOSTAZIONI PAGINA ---
\usepackage{geometry}
\geometry{a4paper, top=2.5cm, bottom=2.5cm, left=2.5cm, right=2.5cm}



% --- TITOLO ---
\title{Appunti di Interpreti e Compilatori}
\author{}
\date{}

\begin{document}

\maketitle
\tableofcontents
\newpage

\include{capitoli/introduzione-traduttori}
\include{capitoli/analisi-lessicale}
\section{Definizione della Sintassi}

Per specificare la sintassi di un linguaggio, si utilizza una notazione largamente impiegata nota come \textbf{grammatica libera dal contesto} (o semplicemente grammatica). Una grammatica descrive in modo naturale la struttura gerarchica della maggior parte dei costrutti dei linguaggi di programmazione.

\subsection{Definizione Formale di una Grammatica}
Una grammatica libera dal contesto è una quadrupla $G = (V, \Sigma, P, S)$ ed è definita da quattro componenti principali:
\begin{enumerate}
    \item Un insieme finito di simboli \textbf{terminali}, $\Sigma$, detti anche \textbf{token}.
    \item Un insieme finito di simboli \textbf{non-terminali}, $V$. Per definizione, $V \cap \Sigma = \emptyset$.
    \item Un insieme finito di \textbf{produzioni}, $P$, nella forma $A \rightarrow \alpha$, dove $A \in V$ è la \textbf{testa} e $\alpha \in (\Sigma \cup V)^*$ è il \textbf{corpo}.
    \item Un simbolo \textbf{iniziale}, $S$, scelto tra i non-terminali ($S \in V$).
\end{enumerate}
Una grammatica è detta "libera dal contesto" perché è possibile applicare una produzione $A \rightarrow \alpha$ indipendentemente da dove si trova il non-terminale $A$.

\subsection{Definizioni Chiave}

\begin{itemize}
    \item \textbf{Derivazione Diretta ($\Rightarrow$):} 
    Rappresenta un singolo passo di riscrittura, dove un non-terminale viene sostituito dal corpo di una sua produzione. 
    \begin{itemize}
        \item Il passaggio da $E$ a $E + T$ è una \textbf{derivazione diretta}, poiché abbiamo applicato la produzione $E \rightarrow E + T$.
        \item Allo stesso modo, $T + T \Rightarrow \mathbf{id} + T$ è una derivazione diretta, dove il primo $T$ è stato sostituito usando la produzione $T \rightarrow \mathbf{id}$.
    \end{itemize}

    \item \textbf{Derivazione in zero o più passi ($\Rightarrow^*$):} 
    Indica una sequenza completa (o parziale) di derivazioni dirette. Poiché è possibile arrivare da $E$ a $\mathbf{id} + T$ in 3 passi, possiamo scrivere:
    \[ E \Rightarrow^* \mathbf{id} + T \]
    La relazione è riflessiva, quindi è corretto anche scrivere $E \Rightarrow^* E$ (zero passi).
    
    \item \textbf{Forma Sentenziale:} 
    È una qualsiasi stringa intermedia nel processo di derivazione che può contenere sia terminali che non-terminali. Nel nostro esempio, tutte le seguenti stringhe sono forme sentenziali:
    \begin{itemize}
        \item $E$ (la forma sentenziale iniziale)
        \item $E + T$
        \item $T + T$
        \item $\mathbf{id} + T$
    \end{itemize}

    \item \textbf{Frase:} 
    È una forma sentenziale che è composta \textit{unicamente} da simboli terminali. Rappresenta una stringa finale e valida del linguaggio.
    \begin{itemize}
        \item La stringa $\mathbf{id} + \mathbf{id}$ è una \textbf{frase}, perché è stata derivata dal simbolo iniziale e contiene solo i simboli terminali $\mathbf{id}$ e $+$.
        \item La stringa $\mathbf{id} + T$, invece, è una forma sentenziale ma \textbf{non} è una frase, perché contiene ancora il non-terminale $T$.
    \end{itemize}

    \item \textbf{Linguaggio Generato ($L(G)$):} 
    È l'insieme di tutte le frasi che una grammatica G può generare. La nostra frase $\mathbf{id} + \mathbf{id}$ è solo uno degli infiniti elementi del linguaggio $L(G)$ definito dalla grammatica d'esempio. Altre frasi sarebbero $\mathbf{id}$, $\mathbf{id} + \mathbf{id} + \mathbf{id}$, ecc.
    
    \item La differenza tra \textbf{derivazione sinistra} e \textbf{derivazione destra} riguarda l'ordine con cui vengono sostituiti i non-terminali in una forma sentenziale. Per una data grammatica, entrambe le derivazioni producono lo stesso albero di parsing, ma lo costruiscono in un ordine diverso.
    \begin{itemize}
        \item \textbf{Derivazione Sinistra (Leftmost Derivation):} A ogni passo della derivazione, viene sempre espanso il \textbf{non-terminale più a sinistra} presente nella forma sentenziale. .
        
        \item \textbf{Derivazione Destra (Rightmost Derivation):} A ogni passo della derivazione, viene sempre espanso il \textbf{non-terminale più a destra}.
    \end{itemize}
\end{itemize}
\subsection{Esempio di Grammatica e Derivazione}
Consideriamo una grammatica per espressioni:
\begin{itemize}
    \item $V = \{E, I\}$
    \item $\Sigma = \{a, b, 0, 1, +, *, (, )\}$
    \item $S = E$
    \item $P = \{ E \rightarrow I \mid E+E \mid E*E \mid (E) \, ; \quad I \rightarrow a \mid b \mid Ia \mid Ib \mid I0 \mid I1 \}$
\end{itemize}

Verifichiamo che la stringa \texttt{ab*(b01+ba)} appartenga al linguaggio generato:
\begin{align*}
E & \Rightarrow E * E \\
  & \Rightarrow I * E \\
  & \Rightarrow Ib * E \Rightarrow ab * E \\
  & \Rightarrow ab * (E) \\
  & \Rightarrow ab * (E + E) \\
  & \Rightarrow ab * (I + E) \\
  & \Rightarrow ab * (I1 + E) \Rightarrow ab * (I01 + E) \Rightarrow ab * (b01 + E) \\
  & \Rightarrow ab * (b01 + I) \\
  & \Rightarrow ab * (b01 + Ia) \Rightarrow ab * (b01 + ba)
\end{align*}

\subsection{Albero di Parsing (Parse Tree)}
Un albero di parsing rappresenta graficamente il modo in cui una stringa del linguaggio
può essere derivata dal simbolo iniziale.
\begin{itemize}
    \item \textbf{Struttura:} La radice è il simbolo iniziale. I nodi interni sono non-terminali. Le foglie sono terminali e, lette da sinistra a destra, formano la frase derivata.
    \item \textbf{Visita:} La frase originale viene fuori facendo una visita in pre-ordine dell'albero, esaminando da sinistra la zona più "profonda".
\end{itemize}
\vspace{0.3 cm}
% --- Rappresentazione testuale dell'albero di parsing ---
\begin{Verbatim}[frame=single, label=Albero di Parsing per ab*(b01+ba)]
                E
                |
        ------------------
       |        |         |
       E        * E
       |                  |
       I                 (E)
       |                  |
   ---------          ---------
  |         |        |    |    |
  I         b        E    +    E
  |                    |         |
  a                    I         I
                       |         |
                   ---------   -----
                  |    |    |   |   |
                  I    0    1   I   a
                  |             |
                  b             b
\end{Verbatim}


\subsection{Correttezza e Completezza}
Una grammatica G è corretta e completa rispetto a un linguaggio L se $L(G) = L$.
\begin{itemize}
    \item \textbf{Correttezza:} Ogni stringa derivabile appartiene al linguaggio ($S \Rightarrow^* w \implies w \in L$).
    \item \textbf{Completezza:} Ogni stringa del linguaggio è derivabile ($w \in L \implies S \Rightarrow^* w$).
\end{itemize}

\vspace{0.2 cm}
\textbf{Dimostrazione di Completezza (per induzione su $|w|$):}
\begin{itemize}
    \item \textbf{Base:} Se $|w|=0$, allora $w=\epsilon$. Se $\epsilon \in L(G)$, allora deve esistere $S \Rightarrow^* \epsilon$.
    \item \textbf{Passo Induttivo:} Supponiamo che per ogni stringa $w$ con $|w| \le n$ valga la tesi. Se $|w|=n+1$, allora $w$ sarà nella forma $u \text{ op } v$ o simile, e per ipotesi induttiva esisteranno le derivazioni per $u$ e $v$.
\end{itemize}

\vspace{0.1 cm}
\textbf{Dimostrazione di Correttezza (induzione sul numero di passi della derivazione):}
\begin{itemize}
    \item \textbf{Base:} Se la derivazione ha 1 passo ($S \Rightarrow^1 w$), si dimostra che $w \in L$. Ad esempio, se le produzioni con un passo sono $S \rightarrow \epsilon \mid 0 \mid 1$, allora le stringhe $\epsilon, 0, 1$ appartengono al linguaggio.
    
    \item \textbf{Passo Induttivo:} Si assume che per ogni derivazione di lunghezza $\le n$ la tesi sia valida. Per una derivazione di $n+1$ passi, si dimostra che anche la stringa risultante appartiene al linguaggio, basandosi sul fatto che è ottenuta da una stringa derivabile in $n$ passi. \\
    Esempio: $S \Rightarrow 0S0 \Rightarrow^* 0x0$. Se per ipotesi induttiva $x \in L$, e la regola di costruzione del linguaggio prevede che $0x0$ sia una stringa valida, allora la tesi è dimostrata.
\end{itemize}


\subsection{Grammatiche Regolari}

Le grammatiche regolari sono un sottoinsieme delle grammatiche libere dal contesto con regole di produzione più restrittive.

\begin{itemize}
    \item \textbf{Produzioni Regolari Destre:} Hanno la forma $X \rightarrow aY$ oppure $X \rightarrow a$, dove $X, Y \in V$ (non-terminali) e $a \in \Sigma$ (un terminale). È permessa anche la produzione $X \rightarrow \epsilon$.
    
    \item \textbf{Produzioni Regolari Sinistre:} Hanno la forma $X \rightarrow Ya$ oppure $X \rightarrow a$.
\end{itemize}

A differenza delle grammatiche libere dal contesto (dove una produzione è $A \rightarrow \alpha$ con $\alpha$ qualsiasi), in una grammatica regolare il non-terminale, se presente, può trovarsi solo all'inizio o alla fine del corpo della produzione, e può essercene al massimo uno.

Una grammatica è regolare se tutte le sue produzioni sono solo regolari destre o solo regolari sinistre.

\subsubsection{Forma delle Frasi}
Le frasi generate da una grammatica regolare sono sempre della forma $S \Rightarrow^* wX$, dove $w \in \Sigma^*$ e $X \in V$ (o $X$ è assente). Questo significa che ad ogni passo viene "consumato" un terminale e il non-terminale (se presente) viene spinto alla fine della forma sentenziale.


\subsubsection{Esempi}

\begin{itemize}

    \item \textbf{Linguaggio delle stringhe di lunghezza pari} \\
    $L = \{w \in \{a,b\}^* \mid |w| \text{ è pari}\}$.
    In questa grammatica, S genera stringhe di lunghezza pari e D genera stringhe di lunghezza dispari.
    \begin{Verbatim}[frame=single]
S -> aD | bD | epsilon
D -> aS | bS
    \end{Verbatim}

    \item \textbf{Linguaggio delle stringhe che contengono 'aaa'} \\
    $L = \{w \in \{a,b\}^* \mid w \text{ contiene la sottostringa 'aaa'}\}$.
    Equivalente all'espressione regolare $(a|b)^*aaa(a|b)^*$.
    \begin{itemize}
        \item S: genera stringhe che non contengono 'aaa' e finiscono con 'b' (o la stringa vuota).
        \item A: genera stringhe che non contengono 'aaa' e finiscono con 'a'.
        \item B: genera stringhe che non contengono 'aaa' e finiscono con 'aa'.
        \item C: genera stringhe che contengono 'aaa'.
    \end{itemize}
    \begin{Verbatim}[frame=single]
S -> bS | aA | epsilon
A -> aB | bS
B -> aC | bS
C -> aC | bC
    \end{Verbatim}

    \item \textbf{Linguaggio delle stringhe che non contengono 'aaa'} \\
    $L = \{w \in \{a,b\}^* \mid w \text{ non contiene la sottostringa 'aaa'}\}$.
    \begin{Verbatim}[frame=single]
S -> aA | bS | epsilon
A -> aB | bS | epsilon
B -> bS | epsilon
    \end{Verbatim}

    \item \textbf{Linguaggio delle stringhe che non iniziano con 'aaa'} \\
    $L = \{w \in \{a,b\}^* \mid w \text{ non ha 'aaa' come prefisso}\}$.
    \begin{Verbatim}[frame=single]
S -> aA | bC | epsilon
A -> aB | bC | epsilon
B -> bC | epsilon
C -> aC | bC | epsilon
    \end{Verbatim}

    \item \textbf{Linguaggio delle stringhe che non contengono 'aba'} \\
     $L = \{w \in \{a,b\}^* \mid w \text{ non contiene la sottostringa 'aba'}\}$.
    \begin{itemize}
        \item S: la stringa finisce con 'b' o e' vuota.
        \item A: la stringa finisce con 'a'.
        \item B: la stringa finisce con 'ab'.
    \end{itemize}
    \begin{Verbatim}[frame=single]
S -> aA | bS | epsilon
A -> aA | bB
B -> aA | bS
    \end{Verbatim}

    \item \textbf{Linguaggio delle stringhe con numero pari di 'a' E numero pari di 'b'} \\
    $L = \{w \in \{a,b\}^* \mid (\#_a(w) \text{ è pari}) \land (\#_b(w) \text{ è pari})\}$.
    \begin{itemize}
        \item S: pari 'a', pari 'b'.
        \item A: dispari 'a', pari 'b'.
        \item B: pari 'a', dispari 'b'.
        \item C: dispari 'a', dispari 'b'.
    \end{itemize}
    \begin{Verbatim}[frame=single]
S -> aA | bB | epsilon
A -> aS | bC
B -> aC | bS
C -> aB | bA
    \end{Verbatim}

    \item \textbf{Linguaggio delle stringhe con numero pari di 'a' O numero dispari di 'b'} \\
    $L = \{w \in \{a,b\}^* \mid (\#_a(w) \text{ è pari}) \lor (\#_b(w) \text{ è dispari})\}$.
    \begin{Verbatim}[frame=single]
S -> aA | bB | epsilon
A -> aS | bC
B -> aC | bS | epsilon
C -> aB | bA | epsilon
    \end{Verbatim}


\end{itemize}
\subsection{Gerarchia di Chomsky}

La \textbf{Gerarchia di Chomsky} riguarda la classificazione delle grammatiche. Si differenziano per il tipo di produzioni.

\begin{enumerate}
    \item \textbf{Grammatiche senza restrizioni (a struttura di frase) - Tipo 0:} Le produzioni hanno la forma $\alpha \rightarrow \beta$, dove $\alpha \in (V \cup \Sigma)^*$ e $\beta \in (V \cup \Sigma)^*$. Ha senso chiedere che $\alpha$ contenga almeno un non terminale: $\alpha \in (V \cup \Sigma)^* V (V \cup \Sigma)^*$.
    \item \textbf{Grammatiche contestuali - Tipo 1:} Le produzioni possono avere queste forme:
    \begin{enumerate}
        \item $\alpha \rightarrow \beta$, dove $\alpha, \beta \in (V \cup \Sigma)^+$, e inoltre $|\alpha| \le |\beta|$.
        \item $\alpha_1 A \alpha_2 \rightarrow \alpha_1 \beta \alpha_2$, con $A \in V$, $\alpha_1, \alpha_2, \beta \in (V \cup \Sigma)^*$, e $\beta \neq \epsilon$.
    \end{enumerate}
    \item \textbf{Grammatiche non contestuali (context-free) - Tipo 2}
    \item \textbf{Grammatiche regolari - Tipo 3}
\end{enumerate}



Un linguaggio formale appartiene a un determinato \textbf{tipo} se è generato da una grammatica di quel tipo. Se un linguaggio è di tipo `i`, ma non esiste nessuna grammatica di tipo `i+1` che lo genera, allora il linguaggio è propriamente di tipo `i`.

I linguaggi generati dalle grammatiche dipendono dal tipo della gerarchia a cui appartengono:

\begin{description}
    \item[Tipo 0 - Grammatiche senza restrizioni:] danno luogo ai cosiddetti \textbf{linguaggi ricorsivamente enumerabili}.
    \item[Tipo 1 - Grammatiche contestuali (context-sensitive):] danno luogo ai \textbf{linguaggi contestuali}.
    \item[Tipo 2 - Grammatiche non contestuali (context-free):] danno luogo ai \textbf{linguaggi non contestuali}.
    \item[Tipo 3 - Grammatiche regolari:] danno luogo ai \textbf{linguaggi regolari}.
\end{description}

Questa classificazione forma una gerarchia di inclusione: ogni linguaggio regolare è anche libero dal contesto, ogni linguaggio libero dal contesto è anche contestuale, e così via.

\subsubsection{Macchine Astratte Riconoscitrici}
Per ogni tipo di linguaggio esiste una macchina astratta (un automa) che serve per stabilire se una data stringa appartiene a quel linguaggio.
\begin{description}
    \item[Tipo 0:] Macchina di Turing
    \item[Tipo 1:] Automa linearmente limitato 
    \item[Tipo 2:] Automa a pila 
    \item[Tipo 3:] Automa a stati finiti 
\end{description}
Un automa di un certo livello può sempre riconoscere i linguaggi dei livelli inferiori. Ad esempio, un automa a pila (Tipo 2) può riconoscere anche i linguaggi regolari (Tipo 3).

\subsection{Esempi di Linguaggi e Grammatiche}
\begin{itemize}
    \item \textbf{Linguaggio $L = \{a^n b^m \mid n, m \ge 0\}$} \\
    Nota: La grammatica $S \rightarrow aSb \mid \epsilon$ genera in realtà $\{a^n b^n\}$. La grammatica corretta per $\{a^n b^m\}$ è:
    \begin{Verbatim}[frame=single]
S -> AB
A -> aA | epsilon
B -> bB | epsilon
    \end{Verbatim}

    \item \textbf{Linguaggio $L = \{w \in \{a,b\}^* \mid |w|_a = |w|_b\}$} (stringhe con lo stesso numero di 'a' e 'b')
    \begin{Verbatim}[frame=single]
S -> epsilon | aSbS | bSaS
    \end{Verbatim}

    \item \textbf{Linguaggio dei palindromi}
    \begin{Verbatim}[frame=single]
S -> epsilon | 0 | 1 | 0S0 | 1S1
    \end{Verbatim}
    
    \item \textbf{Linguaggio $L = \{a^n b^m c^k \mid n, k \ge 0, m > 0\}$}
    \begin{Verbatim}[frame=single]
S -> AC
A -> aA | epsilon
B -> bBC | bC
C -> cC | epsilon
    \end{Verbatim}

    \item \textbf{Linguaggio $L = \{a^n b^n c^n \mid n \ge 0\}$} \\
    Questo linguaggio \textbf{non è libero dal contesto} perché non è possibile "contare" `n` per tutti e tre i simboli con un automa a pila.


    \item \textbf{Linguaggio $L = \{a^n b^n c^k d^k \mid n,m,k,j \ge 0\}$}
    \begin{Verbatim}[frame=single]
S -> AB
A -> aAb | epsilon
B -> cBd | epsilon
    \end{Verbatim}

    \item \textbf{Linguaggio $L = \{a^n b^k c^k d^n \mid n,k \ge 0\}$}
    \begin{Verbatim}[frame=single]
S -> aSd | aAd
A -> bAc | bc
    \end{Verbatim}

    \item \textbf{Linguaggio $L = \{a^n b^k c^{2n+k} \mid n,k \ge 0\}$}
    \begin{Verbatim}[frame=single]
S -> aScc | B
B -> bBc | epsilon
    \end{Verbatim}
    Nota: La grammatica negli appunti sembra avere un errore. Una versione corretta potrebbe essere la precedente.

    \item \textbf{Esempio di Derivazione} \\
    Grammatica per $a^n b^m c^m$:
    \begin{Verbatim}[frame=single]
S -> aSBC | aBC
CB -> BC
bB -> bb
bC -> bc
cC -> cc
    \end{Verbatim}
    Derivazione di `aaabbbccc`:
    \begin{align*}
        S & \Rightarrow aSBC \\
          & \Rightarrow aaSBCBC \\
          & \Rightarrow aaaBCBCBC \quad \text{(Generazione completata: 3a, 3B, 3C)} \\
          & \Rightarrow aaaBBCBCBC \quad \text{(Sposto la prima C a destra: $CB \to BC$)} \\
          & \Rightarrow aaaBBCBCC \quad \text{(Sposto la seconda C: $CB \to BC$)} \\
          & \Rightarrow aaaBBBCCC \quad \text{(Ordinamento completato: $CB \to BC$)} \\
          & \Rightarrow aaabBBCCC \quad \text{(Inizio conversione: $aB \to ab$)} \\
          & \Rightarrow aaabbBCCC \quad \text{(Propago la b: $bB \to bb$)} \\
          & \Rightarrow aaabbbCCC \quad \text{(Propago la b: $bB \to bb$)} \\
          & \Rightarrow aaabbbcCC \quad \text{(La b converte la C: $bC \to bc$)} \\
          & \Rightarrow aaabbbccC \quad \text{(Propago la c: $cC \to cc$)} \\
          & \Rightarrow aaabbbccc \quad \text{(Propago la c: $cC \to cc$)}
        \end{align*}

\end{itemize}


\section{Automi a Stati Finiti}

\subsection{Automa a Stati Finiti Deterministico (DFA)}
Un automa a stati finiti deterministico (DFA) è definito formalmente come una quintupla $A = (Q, \Sigma, \delta, q_0, F)$:
\begin{itemize}
    \item $Q$: un insieme finito di stati.
    \item $\Sigma$: un insieme finito di simboli di input (l'alfabeto).
    \item $\delta$: la funzione di transizione, definita come $\delta: (Q \times \Sigma) \rightarrow Q$. Per una coppia (stato, simbolo) restituisce un singolo stato di arrivo (es. $\delta(q_i, a) = q_j$).
    \item $q_0 \in Q$: lo stato iniziale.
    \item $F \subseteq Q$: l'insieme degli stati finali o di accettazione.
\end{itemize}

\subsection{Linguaggio Accettato e Configurazione}
Il linguaggio accettato da un automa A, denotato con $L(A)$, è l'insieme di tutte le stringhe $w$ che vengono accettate dall'automa.
\[ L(A) = \{ w \in \Sigma^* \mid w \text{ è accettata da A} \} \]
Una \textbf{configurazione istantanea} descrive lo stato dell'automa in un dato momento e consiste in una coppia `[stato attuale, stringa ancora da leggere]`.
\begin{itemize}
    \item Esempio: $[q_i, aw] \rightarrow [q_j, w]$ se $\delta(q_i, a) = q_j$.
\end{itemize}

\subsection{Esempio Pratico: Riconoscimento di "11"}
Consideriamo un automa che riconosce stringhe sull'alfabeto $\{0, 1\}$ che contengono due '1' consecutivi.
\begin{itemize}
    \item $Q = \{q_0, q_1, q_2\}$
    \item $\Sigma = \{0, 1\}$
    \item $q_0$ è lo stato iniziale.
    \item $F = \{q_2\}$
    \item La funzione di transizione $\delta$ è definita dalla seguente tabella:
\end{itemize}

\begin{center}
\begin{tabular}{|c|c|c|}
    \hline
    \textbf{Stato} & \textbf{0} & \textbf{1} \\
    \hline
    $q_0$ & $q_0$ & $q_1$ \\
    \hline
    $q_1$ & $q_0$ & $q_2$ \\
    \hline
    $q_2$ & $q_2$ & $q_2$ \\
    \hline
\end{tabular}
\end{center}

\textbf{Computazione per la stringa $w = 10110$:}
\[ [q_0, 10110] \rightarrow [q_1, 0110] \rightarrow [q_0, 110] \rightarrow [q_1, 10] \rightarrow [q_2, 0] \rightarrow [q_2, \epsilon] \]
Poiché l'automa termina nello stato $q_2$, che è uno stato finale, la stringa $w=10110$ è accettata.

\begin{center}
    \begin{tikzpicture}[shorten >=1pt, node distance=2.5cm, on grid, auto]
        % Stati
        \node[state, initial] (q0) {$q_0$};
        \node[state] (q1) [right=of q0] {$q_1$};
        \node[state, accepting, double] (q2) [right=of q1] {$q_2$};
      
        % Transizioni
        \path[->]
          (q0) edge [loop above] node {0} ()
          (q0) edge node {1} (q1)
          (q1) edge [bend left] node {0} (q0)
          (q1) edge node {1} (q2)
          (q2) edge [loop above] node {0,1} ();
      \end{tikzpicture} 
    \end{center}
    
    \subsection{Esempio Pratico: Riconoscimento di tre a consecutive: "aaa"}
    \begin{center}
    \begin{tikzpicture}[shorten >=1pt, node distance=2.2cm, on grid, auto]
        \node[state, initial] (S1) {S};
        \node[state] (A1) [right=of S1] {A};
        \node[state] (B1) [right=of A1] {B};
        \node[state, accepting, double] (C1) [right=of B1] {C};
      
        \path[->]
          (S1) edge [loop above] node {b} ()
               edge node {a} (A1)
               edge [bend left=55] node {b} (B1)
          (A1) edge node {a} (B1)
               edge [bend left=55] node {b} (S1)
          (B1) edge node {a} (C1)
          (C1) edge [loop right] node {a,b} ();
      \end{tikzpicture}
    \end{center}
      
    \subsection{Esempio Pratico: Riconoscimento di tre a non consecutive: "aaa"}
      \begin{center}
      \begin{tikzpicture}[shorten >=1pt, node distance=2.2cm, on grid, auto]
        \node[state, initial, accepting, double] (S2) {S};
        \node[state, accepting, double] (A2) [right=of S2] {A};
        \node[state, accepting, double] (B2) [right=of A2] {B};
        \node[state] (C2) [right=of B2] {C};
      
        \path[->]
          (S2) edge [loop above] node {b} ()
               edge node {a} (A2)
               edge [bend left=55] node {b} (B2)
          (A2) edge node {a} (B2)
               edge [bend left=55] node {b} (S2)
          (B2) edge node {a} (C2)
          (C2) edge [loop right] node {a,b} ();
      \end{tikzpicture}
    \end{center}
    
    \subsection{Complemento di un Linguaggio}
    \textbf{Teorema:} Se $A = (Q, \Sigma, \delta, q_0, F)$ è un DFA che accetta il linguaggio $L(A)$, allora l'automa $A' = (Q, \Sigma, \delta, q_0, Q-F)$ accetta il linguaggio complemento $\Sigma^* - L(A)$.
    \begin{itemize}
        \item In pratica, per ottenere un automa che accetta il linguaggio complemento, è sufficiente scambiare gli stati finali con quelli non finali.
    \end{itemize}
    
    \subsection{Esempio: Riconoscimento di Stringhe con numero di a  Pari}
    Consideriamo un automa per riconoscere il linguaggio delle stringhe su $\{a,b\}$ con un numero pari di 'a' e un numero pari di 'b'.
    \\
    \begin{minipage}{0.45\textwidth}
        \centering
        \begin{tabular}{|c|c|c|}
        \hline
             & $|w|_a$ & $|w|_b$ \\
        \hline
         S & pari    & pari    \\
        \hline
         A & dispari & pari    \\
        \hline
         B & pari    & dispari \\
        \hline
         C & dispari & dispari \\
        \hline
        \end{tabular}
        \end{minipage}%
        \hfill
        \begin{minipage}{0.52\textwidth}
        \centering
        \begin{tikzpicture}[node distance=2cm, on grid, auto]
          \node[state, initial, accepting, double] (S) {S};
          \node[state] (A) [above right=of S] {A};
          \node[state] (B) [below right=of S] {B};
          \node[state] (C) [right=of B] {C};
        
          \path[->] 
          (S) edge [bend left] node {a} (A)
              edge [bend right] node [swap] {b} (B)
          (A) edge [bend left] node {a} (S)
              edge node {b} (C)
          (B) edge [bend right] node [swap] {a} (C)
              edge [bend right] node [swap] {b} (S)
          (C) edge node {b} (A)
              edge [bend right] node [swap] {a} (B);
        \end{tikzpicture}
        \end{minipage}

\vspace{900pt}

        \subsection{Altri Esempi di DFA}
        %--- Esempio 1: stringa che non inizia con "aaa"
        \textbf{Esempio:} stringa che non inizia con \texttt{aaa}
        
        \begin{tikzpicture}[node distance=2cm, auto]
          \node[state, initial, accepting, double] (S) {S};
          \node[state] (A) [right=of S] {A};
          \node[state] (B) [below=of A] {B};
          \node[state] (C) [left=of B] {C};
          \node[state] (D) [right=of B] {D};
        
          \path[->]
            (S) edge node{a} (A)
                edge [bend left] node{b} (C)
            (A) edge [bend left] node{b} (C)
                edge node{a} (B)
            (C) edge [loop below] node{a,b} ()
                edge node{b} (B)
            (B) edge node{a} (D)
                edge [bend left] node{b} (C)
            (D) edge [loop right] node{a,b} ();
        \end{tikzpicture}
        
        \vspace{1em}
        
        %--- Esempio 2: stringa che contiene "aba"
        \textbf{Esempio:} stringa che contiene \texttt{aba}
        
        \begin{tikzpicture}[node distance=2cm, auto]
          \node[state, initial] (S2) {S};
          \node[state] (A2) [right=of S2] {A};
          \node[state] (B2) [right=of A2] {B};
          \node[state, accepting, double] (C2) [right=of B2] {C};
        
          \path[->]
            (S2) edge node{a} (A2)
                 edge [loop above] node{b} (S2)
            (A2) edge [loop above] node{a} ()
                 edge node{b} (B2)
            (B2) edge node{a} (C2)
                 edge [bend left=40] node{b} (S2)
            (C2) edge [loop above] node{b} ();
        \end{tikzpicture}
        
        \vspace{1em}
        
        %--- Esempio 3: ogni a preceduta o seguita da b
        \textbf{Esempio:} ogni \texttt{a} è preceduta o seguita da \texttt{b}
        
        \begin{tikzpicture}[node distance=2cm, auto]
          \node[state, initial, accepting, double] (S3) {S};
          \node[state] (A3) [right=of S3] {A};
          \node[state, accepting, double] (B3) [below=of S3] {B};
          \node[state] (C3) [right=of A3] {C};
        
          \path[->]
            (S3) edge [loop above] node{b} ()
                 edge node{a} (A3)
                 edge [bend left] node{a} (B3)
            (A3) edge node{a} (C3)
            (B3) edge [loop left] node{b} ()
                 edge node{a} (S3)
            (C3) edge [loop right] node{a,b} ();
        \end{tikzpicture}
        
        \vspace{1em}
        
        %--- Esempio 4: il numero di a e b è pari
        \textbf{Esempio:} il numero di \texttt{a} e di \texttt{b} è pari
        
        \begin{minipage}{0.4\textwidth}
        \centering
        \begin{tabular}{|c|c|c|}
        \hline
             & $|w|_a$ & $|w|_b$ \\
        \hline
         S & pari    & pari    \\
        \hline
         A & dispari & pari    \\
        \hline
         B & pari    & dispari \\
        \hline
         C & dispari & dispari \\
        \hline
        \end{tabular}
        \end{minipage}%
        \hfill
        \begin{minipage}{0.58\textwidth}
        \centering
        \begin{tikzpicture}[node distance=2cm, on grid, auto]
          \node[state, initial, accepting, double] (S4) {S};
          \node[state] (A4) [above right=of S4] {A};
          \node[state] (B4) [below right=of S4] {B};
          \node[state] (C4) [right=of B4] {C};
        
          \path[->]
            (S4) edge node{a} (A4)
                 edge node [swap]{b} (B4)
            (A4) edge [bend left] node{a} (S4)
                 edge node{b} (C4)
            (B4) edge node{a} (C4)
                 edge [bend left] node{b} (S4)
            (C4) edge [bend left] node{a} (B4)
                 edge [bend left] node{b} (A4);
        \end{tikzpicture}
        \end{minipage}
        
\subsubsection{Funzione di Transizione per NFA}
\[ \delta: Q \times (\Sigma \cup \{\epsilon\}) \rightarrow 2^Q \]
Questo significa che per uno stato in $Q$ e un simbolo (o $\epsilon$), la funzione restituisce un \textit{insieme} di possibili stati successivi (indicato da $2^Q$, l'insieme delle parti di Q). 

\subsection{Automi a stati Finiti Non-Deterministici (NFA)}

Un \textbf{automa non-deterministico a stati finiti} (NFA) è la quintupla
\(A=(S,\Sigma,\delta,s_0,F)\) con:
\begin{enumerate}
  \item un insieme finito di stati \(S\);
  \item un insieme di simboli d'ingresso \(\Sigma\) (alfabeto); si assume che la stringa nulla \(\varepsilon\) non sia in \(\Sigma\);
  \item una funzione di transizione \(\delta : S \times (\Sigma \cup \{\varepsilon\}) \to 2^{S}\) che restituisce, per stato e simbolo, un insieme di stati successivi;
  \item uno stato \(s_0 \in S\) (\emph{stato iniziale});
  \item un sottoinsieme \(F \subseteq S\) (\emph{stati di accettazione}).
\end{enumerate}
\vspace{16pt}
Sia NFA che DFA si rappresentano con un \emph{grafo di transizione}: nodi = stati, archi etichettati = transizioni; esiste un arco etichettato \(a\) da \(s\) a \(t\) se e solo se \(t \in \delta(s,a)\).
Per gli NFA valgono queste particolarità:
\begin{enumerate}
  \item uno stesso simbolo può etichettare più archi uscenti dallo stesso stato verso stati diversi;
  \item un arco può essere etichettato con la stringa vuota \(\varepsilon\), da sola o insieme ad altri simboli dell’alfabeto.
\end{enumerate}

% Automa NFA per (a|b)*abb
\begin{center}
    \begin{tikzpicture}[node distance=2cm, auto]
      \node[state, initial] (0) {0};
      \node[state] (1) [right=of 0] {1};
      \node[state] (2) [right=of 1] {2};
      \node[state, accepting, double] (3) [right=of 2] {3};
      \path[->]
        (0) edge [loop above] node {a,b} ()
            edge node {a} (1)
        (1) edge node {b} (2)
        (2) edge node {b} (3);
    \end{tikzpicture}
    \\
    \footnotesize Figura 3.19 \quad Un automa finito non-deterministico che accetta \((a|b)^*abb\).
    \end{center}
    
    Secondo la nostra convenzione per i diagrammi di transizione, il doppio cerchio dello stato 3 indica che si tratta di uno stato finale, cioè uno stato d'accettazione. Si noti che l'unico modo per arrivare dallo stato 0 allo stato 3 richiede di seguire un percorso che rimane per un po' nello stato 0, poi passa attraverso lo stato 1, poi il 2 e infine 3, leggendo \texttt{abb} dalla sequenza d'ingresso. In altre parole, le uniche stringhe che consentono di arrivare allo stato finale sono quelle che terminano per \texttt{abb}.
    
    \subsubsection{Tabelle di transizione}
    Gli NFA sono descritti dalla seguente formula:
    \[\delta : Q \times (\Sigma \cup \{\varepsilon\}) \rightarrow 2^{Q}\]
    Possiamo rappresentare un NFA anche mediante una tabella di transizione, le cui righe corrispondono agli stati e le cui colonne indicano i simboli d'ingresso, più la stringa nulla $\varepsilon$. La casella corrispondente a un dato stato e un dato simbolo d'ingresso indica il valore assunto dalla funzione di transizione.
    
    \begin{center}
    \begin{tabular}{c|c|c|c}
    \textbf{Stato} & \textbf{a} & \textbf{b} & $\varepsilon$ \\
    \hline
    0 & $\{0,1\}$ & $\{0\}$ & $\varnothing$ \\
    1 & $\varnothing$ & $\{2\}$ & $\varnothing$ \\
    2 & $\varnothing$ & $\{3\}$ & $\varnothing$ \\
    3 & $\varnothing$ & $\varnothing$ & $\varnothing$ \\
    \end{tabular}
    \\
    \footnotesize Figura 3.20 \quad Tabella di transizione per l'NFA della Figura 3.19.
    \end{center}
    
    %arrivato a pagina n.132 (libro-it)

\subsection{Logica di simulazione di un NFA}

L'algoritmo per simulare un NFA verifica se una stringa in ingresso viene accettata calcolando iterativamente l'insieme degli stati raggiungibili.

\begin{verbatim}
S = epsilon_closure(s0)
c = nextChar()
while (c != eof) {
    S = epsilon_closure(move(S, c))
    c = nextChar()
}
if (intersezione(S, F) != vuoto) return "sì"
else return "no"
\end{verbatim}

\subsection{Costruzione di un NFA da espressioni regolari}
Per la \textbf{costruzione di un NFA da espressioni regolari} si utilizzano i seguenti scheletri fondamentali:

\begin{itemize}
  \item \textbf{Unione} ($r = s \mid t$):  
    Si crea un nuovo stato iniziale che, tramite transizioni $\varepsilon$, si collega sia al punto di ingresso dell'NFA per $s$ sia a quello per $t$. Gli stati finali dei due rami sono poi connessi tramite $\varepsilon$ a un unico nuovo stato finale comune.
    
    \begin{center}
      \begin{tikzpicture}[shorten >=1pt, node distance=1.5cm, on grid, auto]
        % Stati Start e Finale
        \node (start) {start};
        \node[state] (i) [right=of start] {i};
        \node[state, accepting] (f) [right=6cm of i] {f};
     
        % Blocchi ovali per N(s) e N(t)
        \draw (3, 1.5) ellipse (2.5cm and 0.8cm) node {$N(s)$};
        \draw (3, -1.5) ellipse (2.5cm and 0.8cm) node {$N(t)$};
     
        % Stati interni fittizi per i collegamenti
        \node[state, minimum size=0.6cm] (s_in) at (1.5, 1.5) {};
        \node[state, minimum size=0.6cm] (s_out) at (4.5, 1.5) {};
        \node[state, minimum size=0.6cm] (t_in) at (1.5, -1.5) {};
        \node[state, minimum size=0.6cm] (t_out) at (4.5, -1.5) {};
     
        % Frecce
        \path[->] (start) edge (i);
        \path[->] 
         (i) edge node {$\varepsilon$} (s_in)
         (i) edge node [swap] {$\varepsilon$} (t_in)
         (s_out) edge node {$\varepsilon$} (f)
         (t_out) edge node [swap] {$\varepsilon$} (f);
     \end{tikzpicture}
    \end{center}
  
  \item \textbf{Concatenazione} ($r = st$):  
    I due NFA per $s$ e $t$ vengono “collegati in serie”: lo stato finale di $s$ viene collegato tramite una transizione $\varepsilon$ allo stato iniziale di $t$.
    
    \begin{center}
    \begin{tikzpicture}[node distance=1.6cm, auto]
      \node[state, initial] (i) {i};
      \node[state] (s) [right=of i] {};
      \node[state] (t) [right=of s] {};
      \node[state, accepting, double, right=of t] (f) {f};
      \path[->]
        (i) edge (s)
        (s) edge node[above] {$\varepsilon$} (t)
        (t) edge (f);
    \end{tikzpicture}
    \end{center}
  
  \item \textbf{Chiusura di Kleene} ($r = s^*$):  
    Si aggiungono un nuovo stato iniziale e uno finale. Dal nuovo iniziale si può andare con una $\varepsilon$ sia al nuovo finale (per accettare la stringa vuota), sia verso l'NFA di $s$. Dallo stato finale di $s$ partono due $\varepsilon$-transizioni, una che torna all'inizio di $s$ (per ripetizioni) e una che va al nuovo finale.
    
    % Schema 3: Chiusura di Kleene r = s^* (versione pulita)
    \begin{figure}[h]
      \centering
      \begin{tikzpicture}[shorten >=1pt, node distance=2cm, on grid, auto]
         
         % --- 1. Posizionamento dei Nodi ---
         
         % Nodo Start (etichetta)
         \node (start) {start};
         
         % Nuovo Stato Iniziale (i)
         \node[state] (i) [right=1.5cm of start] {i};
  
         % Stati interni fittizi per N(s) (distanziati maggiormente da i)
         % s_in è l'inizio del vecchio automa, s_out è la fine
         \node[state, minimum size=0.8cm] (s_in) [right=3cm of i] {}; 
         \node[state, minimum size=0.8cm] (s_out) [right=4cm of s_in] {};
  
         % Nuovo Stato Finale (f)
         \node[state, accepting] (f) [right=3cm of s_out] {f};
  
         % --- 2. Disegno dell'Ovale N(s) ---
         % Calcolo il centro tra s_in e s_out per disegnare l'ellisse
         \draw ($(s_in)!0.5!(s_out)$) ellipse (3cm and 1.2cm);
         \node at ($(s_in)!0.5!(s_out)$) {\Large $N(s)$};
  
         % --- 3. Frecce e Transizioni ---
         \path[->] 
          (start) edge (i)
          
          % Ingresso nel sotto-automa
          (i) edge node {$\varepsilon$} (s_in)
          
          % Uscita dal sotto-automa
          (s_out) edge node {$\varepsilon$} (f)
          
          % Loop indietro (Ripetizione): da fine s a inizio s
          (s_out) edge [bend right=60] node [swap] {$\varepsilon$} (s_in)
          
          % Salto in avanti (Stringa vuota/Zero occorrenze): da i a f
          (i) edge [bend right=50] node [swap] {$\varepsilon$} (f);
  
      \end{tikzpicture}
  \end{figure}
\end{itemize}


\subsection*{Costruzione step-by-step di un NFA per \((a|b)^*abb\)}

% NFA per la lettera 'a' (prima parte)
\textbf{a:}\quad
\begin{tikzpicture}[baseline=-0.5ex,node distance=1.8cm, 
    every state/.style={minimum size=16pt}, auto]
  \node[state, initial] (q2) {2};
  \node[state, accepting, double] (q3) [right=of q2] {3};
  \path[->]
    (q2) edge node {a} (q3);
\end{tikzpicture}
\qquad
% NFA per la lettera 'b' (seconda parte)
\textbf{b:}\quad
\begin{tikzpicture}[baseline=-0.5ex,node distance=1.8cm, 
    every state/.style={minimum size=16pt}, auto]
  \node[state, initial] (q4) {4};
  \node[state, accepting, double] (q5) [right=of q4] {5};
  \path[->]
    (q4) edge node {b} (q5);
\end{tikzpicture}


\begin{minipage}[c]{0.49\textwidth}
  \textbf{a|b:}
  
  \begin{tikzpicture}[node distance=1.7cm, on grid, auto]
    \node[state, initial] (q1) {1};
    \node[state] (q2) [above right=1.2cm and 1cm of q1] {2};
    \node[state] (q3) [right=of q2] {3};
    \node[state] (q4) [below right=1.2cm and 1cm of q1] {4};
    \node[state] (q5) [right=of q4] {5};
    \node[state, accepting, double] (q6) [right=3.8cm of q1] {6};
    \path[->]
      (q1) edge node[sloped, above] {$\varepsilon$} (q2)
           edge node[sloped, below] {$\varepsilon$} (q4)
      (q2) edge node {a} (q3)
      (q3) edge node[sloped, above] {$\varepsilon$} (q6)
      (q4) edge node {b} (q5)
      (q5) edge node[sloped, below] {$\varepsilon$} (q6);
    % Evidenziazione delle due sottosequenze (opzionale)
    \usetikzlibrary{calc}
    \draw[dashed, magenta, rounded corners=8pt] ($(q2)+(-0.3,0.5)$) rectangle ($(q3)+(0.3,-0.5)$);
    \draw[dashed, magenta, rounded corners=8pt] ($(q4)+(-0.3,-0.5)$) rectangle ($(q5)+(0.3,0.5)$);
  \end{tikzpicture}
  \end{minipage}
  \hfill
  \begin{minipage}[c]{0.49\textwidth}
    \textbf{(a|b)$^*$:}

    \begin{tikzpicture}[node distance=1.3cm, on grid, auto]
      % Stati
      \node[state, initial] (q0) {0};
      \node[state] (q1) [right=of q0] {1};
      \node[state] (q2) [above right=1.3cm and 1.1cm of q1] {2};
      \node[state] (q3) [right=of q2] {3};
      \node[state] (q4) [below right=1.3cm and 1.1cm of q1] {4};
      \node[state] (q5) [right=of q4] {5};
      \node[state] (q6) [right=4.0cm of q1] {6};
      \node[state, accepting, double] (q7) [right=of q6] {7};
    
      % Transizioni
      \path[->]
        (q0) edge node[above] {$\varepsilon$}(q1)
        (q1) edge[bend left=8] node[above] {$\varepsilon$} (q2)
        (q1) edge[bend right=8] node[below] {$\varepsilon$} (q4)
        (q2) edge node {a} (q3)
        (q3) edge[bend left=8] node[above] {$\varepsilon$} (q6)
        (q4) edge node {b} (q5)
        (q5) edge[bend right=8] node[below] {$\varepsilon$} (q6)
        (q6) edge[bend left=20] node[above] {$\varepsilon$} (q1)
        (q0) edge[bend left=30] node[below] {$\varepsilon$} (q7)
        (q6) edge[bend right=18] node[below] {$\varepsilon$} (q7);
    \end{tikzpicture}
    
  \end{minipage}
  
  \textbf{(a|b)$^*$a:}

\begin{tikzpicture}[node distance=1.7cm, on grid, auto]
  % Stati
  \node[state, initial] (q0) {0};
  \node[state] (q1) [right=of q0] {1};
  \node[state] (q2) [above right=1.2cm and 1cm of q1] {2};
  \node[state] (q3) [right=of q2] {3};
  \node[state] (q4) [below right=1.2cm and 1cm of q1] {4};
  \node[state] (q5) [right=of q4] {5};
  \node[state] (q6) [right=4.0cm of q1] {6};
  \node[state] (q7) [right=of q6] {7};
  \node[state, accepting, double] (q8) [right=of q7] {8};

  % Transizioni automa star
  \path[->]
    (q0) edge node[above] {$\varepsilon$} (q1)
    (q1) edge[bend left=12] node[above] {$\varepsilon$} (q2)
    (q1) edge[bend right=12] node[below] {$\varepsilon$} (q4)
    (q2) edge node {a} (q3)
    (q3) edge[bend left=12] node[above] {$\varepsilon$} (q6)
    (q4) edge node {b} (q5)
    (q0) edge[bend left=30] node[below] {$\varepsilon$} (q7)
    (q5) edge[bend right=12] node[below] {$\varepsilon$} (q6)
    (q6) edge[bend left=18] node[above] {$\varepsilon$} (q1);

  % Uscita dal blocco star verso nuovo stato per 'a'
  \path[->]
    (q6) edge node {$\varepsilon$} (q7)
    (q7) edge node {a} (q8);

\end{tikzpicture}


\textbf{(a|b)$^*$abb:}

\begin{tikzpicture}[node distance=1.7cm, on grid, auto]
  % Stati principali
  \node[state, initial] (q0) {0};
  \node[state] (q1) [right=of q0] {1};
  \node[state] (q2) [above right=1.2cm and 1cm of q1] {2};
  \node[state] (q3) [right=of q2] {3};
  \node[state] (q4) [below right=1.2cm and 1cm of q1] {4};
  \node[state] (q5) [right=of q4] {5};
  \node[state] (q6) [right=4.0cm of q1] {6};
  \node[state] (q7) [right=of q6] {7};
  \node[state] (q8) [right=of q7] {8};
  \node[state] (q9) [right=of q8] {9};
  \node[state, accepting, double] (q10) [right=of q9] {10};

  % Transizioni automa star
  \path[->]
    (q0) edge node[above] {$\varepsilon$} (q1)
    (q1) edge[bend left=12] node[above] {$\varepsilon$} (q2)
    (q1) edge[bend right=12] node[below] {$\varepsilon$} (q4)
    (q2) edge node {a} (q3)
    (q3) edge[bend left=12] node[above] {$\varepsilon$} (q6)
    (q4) edge node {b} (q5)
    (q0) edge[bend left=30] node[below] {$\varepsilon$} (q7)
    (q5) edge[bend right=12] node[below] {$\varepsilon$} (q6)
    (q6) edge[bend left=18] node[above] {$\varepsilon$} (q1);

  % Concatenazione abb finale
  \path[->]
    (q6) edge node {$\varepsilon$} (q7)
    (q7) edge node {a} (q8)
    (q8) edge node {b} (q9)
    (q9) edge node {b} (q10);
\end{tikzpicture}


\subsection{Convertire un NFA in un DFA}

L'idea generale alla base della \textbf{costruzione per sottoinsiemi} è che ogni stato del DFA costruito corrisponda a un \textit{insieme} di stati dell'NFA di partenza. Dopo aver letto una sequenza di input $a_1 a_2 \dots a_n$, il DFA si troverà nello stato che rappresenta l'insieme di tutti i possibili stati in cui l'NFA potrebbe trovarsi processando quella stessa sequenza.

Un aspetto positivo di questo approccio è che, per i linguaggi di interesse pratico, il numero di stati del DFA risultante è approssimativamente lo stesso dell'NFA, evitando l'esplosione esponenziale degli stati che teoricamente potrebbe verificarsi.

\subsubsection*{Funzioni di base}
Per gestire correttamente le $\varepsilon$-transizioni, definiamo le seguenti operazioni sugli stati dell'NFA:

\paragraph{$\varepsilon$-closure(s)}
L'insieme degli stati del NFA raggiungibili da uno stato $s$ seguendo solo percorsi di $\varepsilon$-transizioni (incluso $s$ stesso).
\begin{align*}
  \varepsilon\text{-cl}(6) &= \{6, 7, 1, 2, 4\} \\
  \varepsilon\text{-cl}(8) &= \{8\} \\
  \varepsilon\text{-cl}(0) &= \{0, 1, 2, 4, 7\}
\end{align*}

\paragraph{$\varepsilon$-closure(T)}
Estensione della funzione precedente a un insieme di stati $T$. È l'unione delle $\varepsilon$-closure di tutti gli stati in $T$:
\[
\varepsilon\text{-cl}(T) = \bigcup_{s \in T} \varepsilon\text{-cl}(s)
\]

\paragraph{move(T, a)}
L'insieme degli stati dell'NFA raggiungibili da un qualsiasi stato nell'insieme $T$ consumando il simbolo di input $a$.

\subsubsection*{Costruzione della Tabella di Transizione (Dtran)}
L'algoritmo costruisce gli stati del DFA ($A, B, C, \dots$) e la funzione di transizione \texttt{Dtran}. Lo stato iniziale del DFA è dato da $\varepsilon\text{-closure}(s_0)$, dove $s_0$ è lo stato iniziale dell'NFA.

Calcoliamo lo stato iniziale $A$:
\[
A = \varepsilon\text{-cl}(0) = \{0, 1, 2, 4, 7\}
\]

Ora calcoliamo le transizioni per i simboli di input (es. 'a' e 'b') partendo da $A$:
\begin{align*}
  \mathrm{Dtran}[A, a] &= \varepsilon\text{-cl}(\mathrm{move}(A, a))\\
        &= \varepsilon\text{-cl}(\{3, 8\}) = \{3, 6, 7, 1, 2, 4, 8\} = \{1, 2, 3, 4, 6, 7, 8\} = B \\
  \mathrm{Dtran}[A, b] &= \varepsilon\text{-cl}(\mathrm{move}(A, b))\\
        &= \varepsilon\text{-cl}(\{5\}) = \{1, 2, 4, 5, 6, 7\} = C 
\end{align*}

Avendo scoperto i nuovi stati $B$ e $C$, procediamo a calcolare le loro transizioni:
\begin{align*}
  \mathrm{Dtran}[B, a] &= \varepsilon\text{-cl}(\mathrm{move}(B, a))\\
        &= \varepsilon\text{-cl}(\{3, 8\}) = B
\end{align*}

L'algoritmo prosegue calcolando le transizioni per tutti gli stati scoperti finché non se ne trovano di nuovi. I passaggi rimanenti (omessi per brevità) sono:
\vspace{0.3cm}
\texttt{Dtran[B, b]},\quad \texttt{Dtran[C, a]},\quad \texttt{Dtran[C, b]},\quad \texttt{Dtran[D, a]},\quad \texttt{Dtran[D, b]}

\subsubsection*{Risultato Finale}
Gli stati di accettazione del DFA sono tutti quegli insiemi che contengono almeno uno stato di accettazione dell'NFA originale.

\vspace{0.5cm}
\begin{minipage}{0.45\textwidth}
  \textbf{Tabella delle transizioni DFA}
  
  \begin{tabular}{c|c|c}
   & a & b \\
  \hline
  A & B & C \\
  B & B & D \\
  C & B & E \\
  D & B & D \\
  E & B & C \\
  \end{tabular}
\end{minipage}
\hfill
\begin{minipage}{0.52\textwidth}
  \textbf{Schema DFA}
  
  \begin{tikzpicture}[node distance=1.8cm, auto]
    \node[state, initial] (A) {A};
    \node[state] (B) [right=of A] {B};
    \node[state] (C) [below=of B] {C};
    \node[state] (D) [right=of B] {D};
    \node[state, accepting, double] (E) [below=of D] {E};
  
    % Archi delle transizioni
    \path[->]
      (A) edge[bend left] node {a} (B)
      (A) edge[bend right] node {b} (C)
      (B) edge[loop above] node {a} ()
      (B) edge node {b} (D)
      (C) edge node {a} (B)
      (C) edge[bend right] node {b} (E)
      (D) edge[loop above] node {b} ()
      (D) edge node {a} (B);
  \end{tikzpicture}
\end{minipage}
\subsection{Minimizzazione dei DFA}

L'algoritmo di minimizzazione degli stati si basa sul partizionamento degli stati del DFA in gruppi di stati non distinguibili. Ogni gruppo viene infine fuso in un unico stato del nuovo DFA minimo.

\subsubsection{Concetti Fondamentali}
\begin{itemize}
    \item Data una stringa $x$, si dice che $x$ \textbf{distingue} due stati se, partendo da questi stati e seguendo le transizioni etichettate da $x$, si arriva esattamente in uno stato di accettazione e nell'altro in uno stato non accettante.
    \item Uno stato $A$ è \textbf{distinguibile} da uno stato $B$ se esiste almeno una stringa che li distingue.
    \item Due stati $A$ e $B$ sono \textbf{equivalenti} (o non distinguibili) se nessuna stringa li distingue. In un DFA minimo, stati equivalenti vengono fusi insieme.
\end{itemize}

\subsubsection{Algoritmo}
Si parte da una partizione iniziale $\Pi = \{F, S-F\}$ che separa stati finali da stati non finali. Si raffina la partizione iterativamente finché non è più possibile suddividere i gruppi.
Due stati $s, t$ rimangono nello stesso gruppo se e solo se per ogni simbolo di input $a$, le transizioni $\delta(s, a)$ e $\delta(t, a)$ portano a stati che appartengono allo stesso gruppo nella partizione corrente.

\subsubsection{Esempio Pratico}
Consideriamo il seguente automa (modificato per rendere A e C equivalenti):

\begin{center}
\begin{tikzpicture}[node distance=2.5cm, auto, >=stealth]
    \node[state, initial] (A) {A};
    \node[state] (B) [right=of A] {B};
    \node[state] (C) [above=of B] {C};
    \node[state] (D) [right=of B] {D};
    \node[state, accepting, double] (E) [right=of D] {E};
  
    % Transizioni
    \path[->]
      (A) edge node {a} (B)
          edge [bend left=40] node {b} (E) % A va in E con b (come C)
      (B) edge node {b} (D)
          edge [loop below] node {a} ()
      (C) edge node {a} (B) % C va in B con a (come A)
          edge node {b} (E) % C va in E con b (come A)
      (D) edge [bend left=20] node {b} (E)
          edge [loop above] node {a} ()
      (E) edge [loop right] node {b} ();
\end{tikzpicture}
\end{center}

\textbf{Passi della Minimizzazione:}
\begin{enumerate}
    \item \textbf{Partizione Iniziale:} Separazione tra finali e non finali.
    \[ \Pi_0 = \{ \{A, B, C, D\}, \{E\} \} \]
    
    \item \textbf{Raffinamento 1:} Analizziamo il gruppo $\{A, B, C, D\}$ con l'input 'b'.
    \begin{itemize}
        \item $A \xrightarrow{b} E$ (gruppo 2)
        \item $C \xrightarrow{b} E$ (gruppo 2)
        \item $D \xrightarrow{b} E$ (gruppo 2)
        \item $B \xrightarrow{b} D$ (gruppo 1) $\rightarrow$ \textbf{B si comporta diversamente!}
    \end{itemize}
    Nuova partizione:
    \[ \Pi_1 = \{ \{A, C, D\}, \{B\}, \{E\} \} \]

    \item \textbf{Raffinamento 2:} Analizziamo il gruppo $\{A, C, D\}$ con l'input 'a'.
    \begin{itemize}
        \item $A \xrightarrow{a} B$ (gruppo B)
        \item $C \xrightarrow{a} B$ (gruppo B)
        \item $D \xrightarrow{a} D$ (gruppo A,C,D) $\rightarrow$ \textbf{D si comporta diversamente!}
    \end{itemize}
    Nuova partizione:
    \[ \Pi_2 = \{ \{A, C\}, \{B\}, \{D\}, \{E\} \} \]
    
    \item \textbf{Verifica Finale:} A e C sono equivalenti?
    \begin{itemize}
        \item Con 'a' vanno entrambi in B.
        \item Con 'b' vanno entrambi in E.
    \end{itemize}
    Sì, sono indistinguibili. L'algoritmo termina.
\end{enumerate}

\textbf{Automa Minimo Finale:}
Lo stato iniziale è $\{AC\}$ perché conteneva lo stato iniziale originale.

\begin{center}
  \begin{tikzpicture}[auto, >=stealth, node distance=2.5cm]
    \node[state, initial] (AC) {AC};
    \node[state] (B) [right=of AC] {B};
    \node[state] (D) [right=of B] {D};
    \node[state, accepting, double] (E) [below=of D] {E};
  
    \path[->]
        (AC) edge node {a} (B)
             edge [bend right=20] node {b} (E)
        (B)  edge [loop above] node {a} ()
             edge node {b} (D)
        (D)  edge [loop above] node {a} ()
             edge node {b} (E)
        (E)  edge [loop right] node {b} ();
  \end{tikzpicture}
\end{center}
  

\section{Parsing}
\subsection{Derivazioni e Parsing}
Il \textbf{parsing} è il processo con cui, data una stringa, si verifica se essa appartiene al linguaggio generato da una grammatica, cercando di trovare una derivazione dalla radice (simbolo iniziale) fino alla stringa stessa. Si distinguono:
\begin{itemize}
    \item \textbf{Derivazione sinistra (leftmost):} si espande sempre la non-terminal più a sinistra.
    \item \textbf{Derivazione destra (rightmost):} si espande la più a destra.
\end{itemize}

\begin{tikzpicture}[
    level distance=2.5cm,
    sibling distance=4.2cm,
    grow=down,
    every node/.style={font=\large},
    edge from parent/.style={draw, -}
  ]
  \node {$E$}
    child { node {$T$}
      child { node {$F$}
        child { node {\underline{id}} }
      }
      child { node {$T'$}
        child { node {\underline{$\epsilon$}} }
      }
    }
    child { node {$E'$}
      child { node {\underline{+}} }
      child { node {$T$} [sibling distance=2cm]
        child { node {$F$}
          child { node {\underline{id}} }
        }
        child { node {$T'$} [sibling distance=1.2cm]
          child { node {\underline{*}} }
          child { node {$F$}
            child { node {\underline{id}} }
          }
          child { node {\underline{$\epsilon$}} }
        }
      }
      child { node {$E'$}
        child { node {\underline{$\epsilon$}} }
      }
    };
  \end{tikzpicture}
  
  

\subsection{Parser Top-Down}
Un \textbf{parser top-down} tenta di costruire una derivazione sinistra della stringa d’ingresso:
\begin{itemize}
    \item \textbf{Discesa ricorsiva:} implementazione semplice, tramite chiamate di procedura per ogni non-terminale.
    \item \textbf{Con backtracking:} prova tutte le produzioni possibili; torna indietro in caso di errore.
    \item \textbf{Senza backtracking (LL, tabellare):} per ogni \texttt{(stato, simbolo)} esiste una sola scelta.
\end{itemize}

\vspace{1em}
\begin{algorithm}
    \caption{Procedura di discesa ricorsiva per $A$}
    \begin{algorithmic}[1]
    \Procedure{A}{}
        \State Scegli, per $A$, una produzione $A \rightarrow X_1 X_2 \dots X_k$
        \For{ $i$ da $1$ fino a $k$ }
            \If{ $X_i$ è un non-terminale }
                \State richiama la procedura $X_i()$
            \ElsIf{ $X_i$ è uguale al simbolo d'ingresso corrente $a$ }
                \State procedi al simbolo successivo nella sequenza d'ingresso
            \Else
                \State /* si è verificato un errore */
            \EndIf
        \EndFor
    \EndProcedure
    \end{algorithmic}
    \end{algorithm}
    

\textbf{Esempio: per l'algoritmo della discesa ricorsiva}

Stringa da analizzare: \texttt{id + id * id}

% Tabella delle produzioni (puoi regolare ulteriormente la formattazione)
\begin{table}[ht]
    \centering
    \renewcommand{\arraystretch}{1.5}
    \begin{tabular}{|c|c|c|c|c|c|c|}
    \hline
    \textbf{Non} & \multicolumn{6}{c|}{\textbf{Simbolo d'ingresso}} \\
    \cline{2-7}
    \textbf{terminale} & \texttt{id} & \texttt{+} & \texttt{*} & \texttt{(} & \texttt{)} & \texttt{\$ (fine stringa)} \\
    \hline
    $E$  & $E \rightarrow T E'$     &             &             & $E \rightarrow T E'$ &             &             \\
    \hline
    $E'$ &                         & $E' \rightarrow +T E'$ &             &             & $E' \rightarrow \epsilon$ & $E' \rightarrow \epsilon$ \\
    \hline
    $T$  & $T \rightarrow F T'$     &             &             & $T \rightarrow F T'$  &             &             \\
    \hline
    $T'$ &                         & $T' \rightarrow \epsilon$ & $T' \rightarrow *F T'$ &             & $T' \rightarrow \epsilon$ & $T' \rightarrow \epsilon$ \\
    \hline
    $F$  & $F \rightarrow id$       &             &             & $F \rightarrow (E)$   &             &             \\
    \hline
    \end{tabular}
    \end{table}
\textit{La tabella guida le scelte dell'algoritmo in funzione del simbolo d'ingresso.}

\vspace{0.8em}
Produzioni:
\[
\begin{array}{l}
E \rightarrow T E' \\
T \rightarrow F T' \\
F \rightarrow id \\
T' \rightarrow * F T' \mid \epsilon \\
E' \rightarrow + T E' \mid \epsilon \\
\end{array}
\]

\vspace{0.8em}

Passaggi:
\begin{itemize}
    \item $E \rightarrow T E'$
    \item $T \rightarrow F T'$    \hspace{1em} (eseguo $F \rightarrow id$  $\Rightarrow$ \texttt{match()}, avanzo su "+")
    \item $T' \rightarrow \epsilon$
    \item $E' \rightarrow + T E'$  \hspace{1em} (match "+", avanzo su "id")
    \item $T \rightarrow F T'$     \hspace{1em} (match "id", avanzo su "\texttt{*}")
    \item $T' \rightarrow * F T'$  \hspace{1em} (match "*", avanzo su "id")
    \item $F \rightarrow id$       (match "id", arrivo a fine stringa)
    \item $T' \rightarrow \epsilon$
    \item $E' \rightarrow \epsilon$
\end{itemize}


\textbf{Esempio 2:}
\[
\begin{array}{l}
S \rightarrow cAd \\
A \rightarrow ab \mid a \\
S \rightarrow cAd \to c(ab)d \mid c(a)d
\end{array}
\]

Analisi:
\begin{enumerate}
    \item Input: \texttt{cad}
    \item Provo $A \rightarrow ab$\\
          \texttt{c} (\texttt{match}), \texttt{a} (\texttt{match}),\\
          \textbf{errore}: simbolo d'ingresso "d" non corrisponde a "b" $\Rightarrow$ backtracking
    \item Provo $A \rightarrow a$\\
          \texttt{c} (\texttt{match}), \texttt{a} (\texttt{match}), \texttt{d} (\texttt{match}), successo
\end{enumerate}

Nota: Le indentazioni non sono generalmente usate nell'approccio manuale della discesa ricorsiva.


\subsection{Parser Bottom-Up}
Nel \textbf{parser bottom-up} si parte dalla stringa e si cerca di ricostruire l’albero, risalendo fino al simbolo iniziale:
\begin{itemize}
    \item Ricerca una derivazione destra in maniera inversa (\emph{rightmost derivation in reverse}).
\end{itemize}

\subsection{Ricorsione Sinistra}
Se una grammatica presenta ricorsione sinistra (diretta o indiretta), cioè un non-terminale ha una regola di produzione che inizia con se stesso, la \textbf{discesa ricorsiva non funziona} senza modifiche:
\begin{itemize}
    \item \textbf{Ricorsione sinistra diretta:} Se la grammatica ha una produzione ricorsiva sinistra, ovvero:
        
\[
    S \rightarrow S \alpha \mid \beta
    \]
    
    allora la si può riscrivere eliminando la ricorsione sinistra come:
    
    \[
    \begin{cases}
    S \rightarrow \beta A \\
    A \rightarrow \alpha A \mid \varepsilon 
    \end{cases}
    \]

    Questa trasformazione può essere applicata anche in presenza di più ricorsioni sinistre nel sistema di produzioni.

    \item \textbf{Algoritmo di eliminazione:}
    \begin{enumerate}
        \item Ordina i non-terminali.
        \item Sostituisci produzioni ricorsive.
        \item Elimina la ricorsione immediata con una nuova variabile ausiliaria (A).
    \end{enumerate}

    \item \textbf{Ricorsione sinistra indiretta:} La ricorsione sinistra può essere \textbf{non diretta}.
    \end{itemize}       
    \subsection{Esempio di eliminazione della ricorsione sinistra non immediata}

    Consideriamo la grammatica:
    
    \[
    \begin{array}{l}
    S \rightarrow Aa \mid b \\
    A \rightarrow Ac \mid Sd \mid \varepsilon
    \end{array}
    \]
    
    \textbf{Primo passo:} si sostituiscono le occorrenze di $S$ nelle produzioni di $A$ usando la produzione di $S$, ottenendo:
    
    \[
    \begin{array}{l}
    S \rightarrow Aa \mid b \\
    A \rightarrow Ac \mid Aad \mid bd \mid \varepsilon
    \end{array}
    \]
    (Nota: sostituendo $S \to Aa$ in $Sd$, otteniamo $Aad$).
    
    \textbf{Secondo passo:} si identificano le parti ricorsive ($\alpha$) e le parti non ricorsive ($\beta$) per evidenziare la ricorsione sinistra diretta:
    
    \[
    \begin{array}{l}
    \text{Produzioni di } A: A \rightarrow A(c) \mid A(ad) \mid bd \mid \varepsilon \\
    \alpha_1 = c, \quad \alpha_2 = ad \\
    \beta_1 = bd, \quad \beta_2 = \varepsilon
    \end{array}
    \]
    
    \textbf{Terzo passo:} si elimina la ricorsione sinistra diretta da $A$ introducendo la variabile ausiliaria $A'$ e applicando la regola $A \to \beta A'$ e $A' \to \alpha A' \mid \varepsilon$:
    
    \[
    \begin{array}{l}
    S \rightarrow Aa \mid b \\
    A \rightarrow bd A' \mid A' \\
    A' \rightarrow c A' \mid ad A' \mid \varepsilon
    \end{array}
    \]
    (Nota: la produzione $A \to A'$ deriva da $\beta_2 A'$ dove $\beta_2 = \varepsilon$).
    
    In questo modo, la grammatica risultante è \textbf{priva di ricorsione sinistra} e adatta per un parser con discesa ricorsiva.

    Esiste un algoritmo sistematico per eliminare la ricorsione sinistra dalle grammatiche:


    \begin{algorithm}
        \caption{Eliminazione della ricorsione sinistra}
        \begin{algorithmic}[1]
        \State Ordina arbitrariamente i non-terminali come $A_1, A_2, \ldots, A_n$
        \For{ogni $i$ da $1$ fino a $n$}
            \For{ogni $j$ da $1$ fino a $i-1$}
                \State Sostituisci ogni produzione nella forma $A_i \rightarrow A_j \gamma$\\
                \hspace{1.5em}  con le produzioni $A_i \rightarrow \delta_1 \gamma \;|\; \delta_2 \gamma \;|\; \cdots \;|\; \delta_k \gamma$,\\
                \hspace{1.5em}  in cui $A_j \rightarrow \delta_1 \;|\; \delta_2 \;|\; \cdots \;|\; \delta_k$ sono tutte le produzioni per il non-terminale $A_j$ in esame
            \EndFor
            \State Elimina la ricorsione sinistra immediata dalle produzioni per $A_i$
        \EndFor
        \end{algorithmic}
    \end{algorithm}     

\textbf{Esempio 1:}
\[
\begin{array}{l}
A \rightarrow Ba \mid d \\
B \rightarrow Bb \mid Ac \mid \varepsilon \\
\end{array}
\]

\vspace{1em}
Dopo $i = 2$ (sostituzione di $A \rightarrow Ba$):
\[
\begin{array}{l}
A \rightarrow (Bb \mid Ac \mid \varepsilon) a \mid d \\[0.3em]
= Ba c \mid B b a \mid d a \mid d \\
B \rightarrow Bb \mid Ac \mid \varepsilon \\
\end{array}
\]

\vspace{1em}
Eliminando la ricorsione sinistra su $B$:

\[
\begin{array}{l}
A \rightarrow Ba \mid d \\
B \rightarrow dcB' \mid B' \\
B' \rightarrow bb' \mid acB' \mid \varepsilon \\
\end{array}
\]

\textbf{Esempio 2:}

\[
\begin{array}{l}
A \rightarrow Ba \mid C \\
B \rightarrow Cc \mid A \\
C \rightarrow Cc \mid d \\
\end{array}
\]

\vspace{1em}
Dopo sostituzioni:
\[
\begin{array}{l}
A \rightarrow Ba \mid C \\
B \rightarrow Cc \mid Ba \mid C \\
C \rightarrow Cc \mid d \\
\end{array}
\]

\vspace{1em}
Eliminando la ricorsione sinistra ($B$):

\[
\begin{array}{l}
A \rightarrow Ba \mid C \\
B \rightarrow CbB' \mid CB' \\
B' \rightarrow aB' \mid \varepsilon \\
C \rightarrow Cc \mid d \\
\end{array}
\]

\vspace{1em}
Eliminando la ricorsione sinistra ($C$):

\[
\begin{array}{l}
A \rightarrow Ba \mid C \\
B \rightarrow CbB' \mid CB' \\
B' \rightarrow aB' \mid \varepsilon \\
C \rightarrow dC' \\
C' \rightarrow cC' \mid \varepsilon \\
\end{array}
\]

\subsection{Fattorizzazione comune:}
Se $A \rightarrow \alpha\beta_1 \mid \alpha\beta_2 \mid \ldots \mid \alpha\beta_k \mid \gamma_1 \mid \gamma_2 \mid \ldots \mid \gamma_n$ \\
con $\alpha$ prefisso più lungo comune, allora si può riscrivere, raccogliendo per $\alpha$, come:

\[
\begin{aligned}
A &\rightarrow \alpha A' \mid \gamma_1 \mid \gamma_2 \mid \ldots \mid \gamma_n \\
A' &\rightarrow \beta_1 \mid \beta_2 \mid \ldots \mid \beta_k 
\end{aligned}
\]

\textbf{Esempio 1:}
\[
\begin{aligned}
A &\rightarrow abc d \mid abc e \mid abf \\
\end{aligned}
\]

\vspace{1em}
Fattorizzazione:
\[
\begin{aligned}
A &\rightarrow abc A' \mid abf \\
A' &\rightarrow d \mid e
\end{aligned}
\]

% oppure con doppio passaggio, come nell'immagine:
Oppure:
\[
\begin{aligned}
A &\rightarrow ab A'' \\
A'' &\rightarrow cA' \mid f \\
A' &\rightarrow d \mid e
\end{aligned}
\]


\textbf{Esempio 2: fattorizzazione comune + ricorsione diretta}

\[
\begin{aligned}
A &\rightarrow Abc d \mid Abc e \mid abf
\end{aligned}
\]

Fattorizzazione:
\[
\begin{aligned}
A &\rightarrow Abc A' \mid abf \\
A' &\rightarrow d \mid e
\end{aligned}
\]

Eliminazione ricorsione diretta:
\[
\begin{aligned}
A &\rightarrow abf A'' \\
A'' &\rightarrow bcA' A'' \mid \varepsilon \\
A' &\rightarrow d \mid e
\end{aligned}
\]

% oppure come nell'altra variante dell'immagine:
Oppure:
\[
\begin{aligned}
A &\rightarrow Abc d \mid Abc e \mid abf \\
\rightarrow abf A' \\
A' &\rightarrow bc d A'' \mid bc e A'' \mid \varepsilon \\
A'' &\rightarrow abf \mid cA'
\end{aligned}
\]



\subsection{Schema comparativo dei parser}

% Esempio di tabella comparativa
\begin{table}[h!]
\centering
\begin{tabular}{l|l|l}
Parser & Tecnica & Pro/Contro \\
\hline
Discesa ricorsiva & Top-down & Semplice, non sempre applicabile\\
LL(1), tabellare  & Top-down & Rapido, niente backtracking\\
Bottom-up (LR)    & Bottom-up & Più potente, tabellare\\
\end{tabular}
\end{table}

\subsection{FIRST e FOLLOW nelle Grammatiche Formali}
\subsection*{Definizione di FIRST}
\begin{itemize}
    \item L'insieme \textbf{FIRST} di un simbolo grammaticale (o stringa di simboli) indica i terminali che possono essere il primo simbolo derivato da quella variabile o sequenza.
    \item \textbf{Regole essenziali:}
    \begin{itemize}
        \item Se il simbolo è un terminale, \(\mathrm{FIRST}\) è il simbolo stesso.
        \item Se il simbolo è una variabile, si analizzano le produzioni possibili. Per ogni produzione, si aggiungono i simboli \(\mathrm{FIRST}\) derivabili dalle sottostringhe secondo le regole specifiche.
        \item Per una stringa \(X_1X_2\ldots X_n\) si applica una regola iterativa: si prende il \(\mathrm{FIRST}\) del primo simbolo non nullo e, se necessario, anche il \(\mathrm{FIRST}\) dei successivi se includono la produzione \(\varepsilon\) (vuota).
    \end{itemize}
\end{itemize}

\subsection*{Esempio di FIRST}
\begin{verbatim}
S → Ax | yB
B → zB
A → Ba | S
\end{verbatim}
\[
\mathrm{FIRST}(B) = \{z\}
\]
\[
\mathrm{FIRST}(A) = \{z, a\}
\]
\[
\mathrm{FIRST}(S) = \{y, x, z, a\}
\]

\subsection*{Definizione di FOLLOW}
\begin{itemize}
    \item L'insieme \textbf{FOLLOW} di una variabile raccoglie tutti i terminali che possono seguire quella variabile in qualche derivazione, includendo anche il marcatore di fine stringa se la variabile può essere l’ultima nel processo di derivazione.
    \item \textbf{Regole essenziali:}
    \begin{itemize}
        \item Si aggiunge il marcatore ``\$'' (fine stringa) a FOLLOW del simbolo di partenza.
        \item Se una variabile \(X\) può essere seguita da una stringa \(Y\), si aggiunge \(\mathrm{FIRST}(Y)\) a FOLLOW\((X)\).
        \item Se \(X\) è alla fine di una produzione, si aggiunge il FOLLOW della variabile precedente (Testa della Produzione).
    \end{itemize}
\end{itemize}

\subsection*{Esempio di FOLLOW}
\begin{verbatim}
S → ACB | Cbb | Ba
A → da | BC
B → g
C → h
\end{verbatim}
\[
\mathrm{FOLLOW}(S) = \{\$\}
\]
\[
\mathrm{FOLLOW}(A) = \{h, g, \$\}
\]
\[
\mathrm{FOLLOW}(B) = \{a, \$, h, g\}
\]
\[
\mathrm{FOLLOW}(C) = \{b, \$, g, h\}
\]


\subsection*{Calcolo di FIRST per una stringa}

Possiamo ora calcolare \textbf{FIRST} per una generica stringa \( X_1X_2\ldots X_n \) come segue. Si aggiungono a \(\mathrm{FIRST}(X_1X_2\cdots X_n)\) tutti i simboli di \(\mathrm{FIRST}(X_1)\) diversi da \( \varepsilon \).
Inoltre,
\begin{itemize}
    \item se \( \varepsilon \in \mathrm{FIRST}(X_1) \), si aggiungono a \(\mathrm{FIRST}(X_1X_2\cdots X_n)\) tutti i simboli (diversi da \( \varepsilon \)) di \(\mathrm{FIRST}(X_2)\);
    \item se \( \varepsilon \in \mathrm{FIRST}(X_2) \), si aggiungono tutti i simboli (diversi da \( \varepsilon \)) di \(\mathrm{FIRST}(X_3) \), e così via.
    \item Infine si aggiunge \(\varepsilon\) a \(\mathrm{FIRST}(X_1X_2\cdots X_n)\) se e solo se \(\varepsilon \in \mathrm{FIRST}(X_i)\) per ogni \( i \).
\end{itemize}

\subsection*{Regole per il calcolo di FOLLOW}
Per calcolare \(\mathrm{FOLLOW}(A)\) per tutti i non-terminali \(A\) si procede applicando le regole seguenti finché non sia più possibile aggiungere nulla all'insieme \textsf{FOLLOW}.

\begin{enumerate}
    \item Si aggiunga \$ a \(\mathrm{FOLLOW}(S)\), ricordando che \(S\) è il simbolo iniziale e \$ è il marcatore di fine della stringa d'ingresso.
    \item Se esiste una produzione del tipo \(A \rightarrow \alpha B \beta\), allora si aggiunga a \(\mathrm{FOLLOW}(B)\) ogni elemento di \(\mathrm{FIRST}(\beta)\) eccetto \( \varepsilon \).
    \item Se esiste una produzione del tipo \(A \rightarrow \alpha B\) oppure del tipo \(A \rightarrow \alpha B \beta\) per cui \(\mathrm{FIRST}(\beta)\) contiene \( \varepsilon \), allora tutti i simboli in \(\mathrm{FOLLOW}(A)\) appartengono anche a \(\mathrm{FOLLOW}(B)\).
\end{enumerate}

\subsection*{Esempio}
Facendo riferimento alla grammatica
\[
\begin{array}{rl}
E   &\rightarrow\ T\ E' \\
E'  &\rightarrow\ +\ T\ E'\ \mid\ \varepsilon \\
T   &\rightarrow\ F\ T' \\
T'  &\rightarrow\ *\ F\ T'\ \mid\ \varepsilon \\
F   &\rightarrow\ (E)\ \mid\ \mathrm{id} \\
\end{array}
\]
possiamo affermare:
\begin{enumerate}
  \item $\mathrm{FIRST}(F) = \mathrm{FIRST}(T) = \mathrm{FIRST}(E) = \{ (,\, \mathrm{id}  \}$.
  Per capire il perché, si noti che le due produzioni per $F$ hanno corpi che iniziano con i due simboli terminali $\mathrm{id}$ e parentesi aperta. $T$ ha una sola produzione il cui corpo inizia con $F$. Dato che $F$ non deriva $\varepsilon$, $\mathrm{FIRST}(T)$ deve coincidere con $\mathrm{FIRST}(F)$. Si ragiona poi allo stesso modo per $\mathrm{FIRST}(E)$.

  \item $\mathrm{FIRST}(E') = \{+,\, \varepsilon\}$. La ragione è che il corpo di una delle due produzioni per $E'$ inizia con il terminale $+$ e il corpo dell’altra è $\varepsilon$. Ogni volta che un non-terminale deriva $\varepsilon$, dobbiamo aggiungere $\varepsilon$ all’insieme $\mathrm{FIRST}$ di quel non-terminale.

  \item $\mathrm{FIRST}(T') = \{\ast,\, \varepsilon\}$. Il ragionamento è simile a quello visto per $\mathrm{FIRST}(E')$.

  \item $\mathrm{FOLLOW}(E) = \mathrm{FOLLOW}(E') = \{\text{)},\, \$\}$. Dato che $E$ è il simbolo iniziale, $\mathrm{FOLLOW}(E)$ deve contenere il simbolo speciale $\$$. La presenza del corpo $(E)$ spiega il perché la parentesi chiusa appartenga a $\mathrm{FOLLOW}(E)$. Per quanto riguarda $E'$, si nota che questo non-terminale appare sempre alla fine dei corpi delle produzioni per $E$. Ne consegue che $\mathrm{FOLLOW}(E')$ deve coincidere con $\mathrm{FOLLOW}(E)$.

  \item $\mathrm{FOLLOW}(T) = \mathrm{FOLLOW}(T') = \{+,\,\text{)},\, \$\}$. Si noti che $T$ appare nei corpi delle varie produzioni sempre seguito da $E'$. Ne consegue che ogni simbolo, a esclusione di $\varepsilon$, che appartiene a $\mathrm{FIRST}(E')$ deve appartenere anche a $\mathrm{FOLLOW}(T)$. Questo ragionamento spiega la presenza del simbolo $+$. Tuttavia, dato che $\mathrm{FIRST}(E')$ contiene $\varepsilon$ (cioè $E' \Rightarrow^* \varepsilon$) ed $E'$ è tutto ciò che segue $T$ nei corpi delle produzioni per $E$, tutti i simboli in $\mathrm{FOLLOW}(E)$ devono anche appartenere a $\mathrm{FOLLOW}(T)$. Questo spiega i simboli $\$$ e parentesi chiusa. Per quanto riguarda $T'$, si osserva che appare solo alla fine delle produzioni per $T$, si conclude che $\mathrm{FOLLOW}(T') = \mathrm{FOLLOW}(T)$.

  \item $\mathrm{FOLLOW}(F) = \{+,\, *,\,\text{)},\, \$\}$. Il ragionamento da seguire è analogo a quello per $T$, svolto al punto precedente.
\end{enumerate}

\subsection{Grammatiche LL(1)}

\subsection*{Definizione}
Una grammatica è detta \textbf{LL(1)} se:
\begin{itemize}
    \item Esiste sempre la possibilità di costruire un parser \textbf{top-down predittivo deterministico}, cioè un parser a discesa ricorsiva senza backtracking.
    \item La prima \textbf{L} indica che si analizza la stringa in ingresso da \emph{sinistra} verso destra (\textit{Left}).
    \item La seconda \textbf{L} indica che si costruisce una \emph{derivazione sinistra} (\textit{Leftmost derivation}).
    \item Il numero \textbf{1} indica che la regola da scegliere viene determinata guardando un solo simbolo \textit{lookahead}, cioè il prossimo simbolo della stringa in input.
\end{itemize}

\subsection*{Condizioni formali per le grammatiche LL(1)}

Una grammatica $G$ è \textbf{LL(1)} se e solo se soddisfa le seguenti condizioni per ogni variabile $A$:

Siano $A \rightarrow \alpha_1 \mid \alpha_2 \mid \dots \mid \alpha_k$ le produzioni per $A$, allora:

\begin{itemize}
    \item $\mathrm{FIRST}(\alpha_i) \cap \mathrm{FIRST}(\alpha_j) = \emptyset \quad\forall\, i \neq j$
    \item Se $\exists\, i$ tale che $\alpha_i \Rightarrow^* \varepsilon$, allora:
    \begin{itemize}
        \item $\alpha_j \not\Rightarrow^* \varepsilon \quad \forall\, j \neq i$
        \item $\mathrm{FOLLOW}(A) \cap \mathrm{FIRST}(\alpha_i) = \emptyset$
    \end{itemize}
\end{itemize}

\vspace{1em}

In modo equivalente, per ogni variabile:
\begin{itemize}
    \item Gli insiemi \textbf{FIRST} relativi alle parti destre delle produzioni sono due a due disgiunti.
    \item Esiste al più una parte destra che può derivare $\varepsilon$ e in questo caso l’insieme \textbf{FOLLOW} della variabile deve essere disgiunto dagli insiemi \textbf{FIRST} di tutte le parti destre, cioè dal \textbf{FIRST} della variabile.
\end{itemize}


\subsection*{Parsing a discesa ricorsiva per grammatiche LL(1)}

Il parsing a discesa ricorsiva LL(1) si implementa tipicamente come una funzione ricorsiva per ogni variabile non terminale. Dato $\alpha_i = X_1 X_2 \ldots X_n$, il codice generale è:

\begin{verbatim}
for (i = 1; i <= n; i++) {
    if (Xi è una variabile)
        Xi();
    else if (Xi == a)
        a = next.token;
    else
        errore();
}
\end{verbatim}


Dove:
\begin{itemize}
    \item $V$ è l'insieme dei simboli non terminali della grammatica (variabili).
    \item $a$ è il simbolo corrente dell’input (lookahead token).
    \item \texttt{next.token} restituisce il prossimo simbolo dell’input.
\end{itemize}

Questo schema viene adattato per ciascuna produzione della grammatica LL(1) durante la scrittura manuale di parser ricorsivi predittivi, consentendo di riconoscere la struttura dell'input senza backtracking.

\subsection*{Note Finali}
\begin{itemize}
    \item In presenza di conflitti (ricorsione sinistra, prefissi comuni), occorre riformulare la grammatica per adattarla a LL(1).
    \item Le grammatiche LL(1) sono fondamentali per costruire parser efficienti, semplici e deterministici.
\end{itemize}


\subsection{Tabelle di parsing predittivo LL(1)}

Le informazioni derivate dagli insiemi \textbf{FIRST} e \textbf{FOLLOW} di una grammatica possono essere raccolte in una \textbf{tabella di parsing predittivo} $M$. In questa tabella, le righe corrispondono alle variabili non terminali della grammatica, e le colonne ai terminali (più il marcatore di fine stringa $\$$). L’elemento $M[A, a]$ indica quale produzione applicare per espandere la variabile $A$ quando il simbolo corrente in ingresso è $a$.

\subsection*{Costruzione della tabella}
Per ogni produzione $A \rightarrow \alpha$ della grammatica $G$:
\begin{enumerate}
    \item Per ogni terminale $a \in \mathrm{FIRST}(\alpha)$, si mette $A \rightarrow \alpha$ in $M[A,a]$.
    \item Se $\varepsilon \in \mathrm{FIRST}(\alpha)$, allora per ogni simbolo $b \in \mathrm{FOLLOW}(A)$ (incluso eventualmente $\$$) si mette $A \rightarrow \alpha$ in $M[A,b]$.
    \item Se in $M[A, a]$ non c’è nessuna regola, si ha una condizione di errore: il simbolo $a$ non può essere derivato dalle produzioni di $A$.
    \item Se $M[A, a]$ contiene più di una regola allora la grammatica non è LL(1) (conflitto tra insiemi $\mathrm{FIRST}$ oppure $\mathrm{FOLLOW}$ e $\mathrm{FIRST}$).
\end{enumerate}

\subsection*{Esempio di tabella di parsing}

Per la grammatica
\[
\begin{array}{rl}
E   &\rightarrow\ T\ E' \\
E'  &\rightarrow\ +T E' \mid \varepsilon \\
T   &\rightarrow\ F T' \\
T'  &\rightarrow\ * F T' \mid \varepsilon \\
F   &\rightarrow\ (E) \mid \mathrm{id}
\end{array}
\]

Tabella di parsing LL(1):
\[
\begin{array}{c|cccccc}
    & \mathrm{id} & + & * & ( & ) & \$ \\
\hline
E   & E \rightarrow T E' &   &   & E \rightarrow T E' &   &   \\
E'  &   & E' \rightarrow +T E' &   &   & E' \rightarrow \varepsilon & E' \rightarrow \varepsilon \\
T   & T \rightarrow F T' &   &   & T \rightarrow F T' &   &   \\
T'  &   & T' \rightarrow \varepsilon & T' \rightarrow * F T' &   & T' \rightarrow \varepsilon & T' \rightarrow \varepsilon \\
F   & F \rightarrow \mathrm{id} &   &   & F \rightarrow (E) &   &   \\
\end{array}
\]

\subsection{Parsing predittivo non ricorsivo}
\begin{itemize}
    \item Lo stack iniziale contiene $\$$ (fondo) e il simbolo iniziale della grammatica.
    \item A ogni passo, il parser esamina il simbolo $X$ in cima allo stack e il simbolo di ingresso corrente $a$:
    \begin{itemize}
      \item \textbf{while} $X \neq \$\ $:
      \begin{itemize}
        \item \textbf{if} $(X = a)$ allora avanza il puntatore $ip$;
        \item \textbf{else if} $(X \in \Sigma \cup \{\$\})$ allora \textit{errore()};
        \item \textbf{else if} $(M[X, a] = \emptyset)$ allora \textit{errore()};
        \item \textbf{else if} $M[X, a] = X \to Y_1 Y_2 \ldots Y_k$ allora:
        \begin{itemize}
          \item produci come uscita $X \to Y_1 Y_2 \ldots Y_k$;
          \item inserisci $Y_k, Y_{k-1}, \ldots, Y_1$ nello stack (con $Y_1$ in cima);
        \end{itemize}
        \item assegna a $X$ il simbolo in cima allo stack $(X = \mathrm{pop}(P))$;
      \end{itemize}
      \item \textbf{if} $(a = \$)$ accetta; altrimenti \textit{errore()}.
    \end{itemize}
    \item Se lo stack contiene solo $\$$ e anche l’input contiene $\$$, l’input viene accettato.
\end{itemize}

\subsection*{Esempio di parsing passo-passo}
\begin{table}[h!]
  \centering
  \caption*{Parsing predittivo non ricorsivo (input: \texttt{id + id * id \$})}
  \renewcommand{\arraystretch}{1.1}
  \begin{tabular}{|l|l|l|l|}
  \hline
  \textbf{Riconosciuta} & \textbf{Stack}   & \textbf{Input}         & \textbf{Azione} \\ \hline
                        & $E\$$            & $id\ +\ id\ *\ id\$ $  &                 \\ \hline
                        & $TE'\$$           & $id\ +\ id\ *\ id\$ $  & output $E \to TE'$ \\ \hline
                        & $FT'E'\$$          & $id\ +\ id\ *\ id\$ $  & output $T \to FT'$ \\ \hline
  $id$                  & $T'E'\$$          & $+ id\ *\ id\$ $       & consuma $id$     \\ \hline
  $id$                  & $E'\$$            & $+ id\ *\ id\$ $       & output $T' \to e$ \\ \hline
  $id$                  & $+TE'\$$          & $+ id\ *\ id \$ $      & output $E' \to +TE'$ \\ \hline
  $id\ +$               & $TE'\$$           & $id\ *\ id\$ $         & consuma $+$      \\ \hline
  $id\ +$               & $FT'E'\$$          & $id\ *\ id\$ $         & output $T \to FT'$ \\ \hline
  $id\ + id$            & $T'E'\$$          & $*\ id\$ $             & consuma $id$     \\ \hline
  $id\ + id$            & $*FT'E'\$$        & $*\ id\$ $             & output $T' \to *FT'$ \\ \hline
  $id\ + id\ *$         & $FT'E'\$$         & $id\$ $                & consuma $*$      \\ \hline
  $id\ + id\ *$         & $idT'E'\$$        & $id\$ $                & output $F \to id$ \\ \hline
  $id\ + id\ *\ id$     & $T'E'\$$          & $\$$                   & consuma $id$     \\ \hline
  $id\ + id\ *\ id$     & $E'\$$            & $\$$                   & output $T' \to e$ \\ \hline
  $id\ + id\ *\ id$     & $ \$ $            & $\$$                   & output $E' \to e$ \\ \hline
  \end{tabular}
  
  \vspace{1ex}
  \textbf{NOTA:} $e$ indica la produzione vuota, cioè $e \equiv \varepsilon$
  \end{table}
  

\begin{table}[h!]
  \centering
  \caption*{Parsing predittivo non ricorsivo (input: \texttt{id * + id \$})}
  \begin{tabular}{|l|l|l|l|}
  \hline
  \textbf{Riconosciuta} & \textbf{Stack}      & \textbf{Input}     & \textbf{Azione} \\ \hline
                        & $E\$$             & $id\ *\ +\ id\ \$ $ &                \\ \hline
                        & $TE'\$$          & $id\ *\ +\ id\ \$ $ & output $E \rightarrow TE'$  \\ \hline
                        & $FTE'\$$         & $id\ *\ +\ id\ \$ $ & output $T \rightarrow FT'$  \\ \hline
                        & $idT'E'\$$       & $id\ *\ +\ id\ \$ $ & output $F \rightarrow id$   \\ \hline
  $id$                  & $T'E'\$$         & $*\ +\ id\ \$ $     & consuma $id$               \\ \hline
  $id$                  & $*FTE'\$$        & $*\ +\ id\ \$ $     & output $T' \rightarrow *FT'$ \\ \hline
  $id\ *$               & $FTE'\$$         & $+\ id\ \$ $        & consuma $*$                \\ \hline
  $id\ *$               & $FTE'\$$         & $+\ id\ \$ $        &                            \\ \hline
  \end{tabular}
  \end{table}
  
  \vspace{1ex}

  Questo tipo di tabella e parsing consente di implementare facilmente parser deterministici, garantendo l’assenza di backtracking quando la grammatica è LL(1).

\section*{Analisi Sintattica Bottom-Up}

I metodi bottom-up costruiscono l’albero di parsing partendo dalle foglie e procedendo verso l’alto fino alla radice.

\vspace{0.5em}
Esempio di grammatica:

\[
E \to T \mid E + T \quad,\quad
T \to F \mid T * F \quad,\quad
F \to id \mid (E)
\]

\vspace{1em}
Sequenze di parsing (esempio con simboli):

\begin{figure}[h]
    \centering
    % Configurazione stile alberi
    \tikzset{
        level distance=20pt,      % Distanza verticale tra i livelli
        sibling distance=5pt,     % Distanza orizzontale tra i nodi
        every node/.style={inner sep=1pt, font=\small},
        edge from parent/.style={draw, thin}
    }

    % --- Passo 1: Stringa iniziale (Livello 0) ---
    \begin{tikzpicture}[baseline=0cm]
        \node at (0,0) {\textbf{id} $*$ \textbf{id}};
        % Rettangolo invisibile per mantenere l'altezza coerente con gli altri
        \path (0,3); 
    \end{tikzpicture}
    \hfill
    % --- Passo 2: Riduzione id -> F (F sale a livello 1) ---
    \begin{tikzpicture}[baseline=0cm]
        % Albero F posizionato a y=20pt (Livello 1)
        \node at (0, 20pt) {F}
            child {node {\textbf{id}}};
        \node [right=0.2cm] at (0,0) {$*$ \textbf{id}};
        \path (0,3);
    \end{tikzpicture}
    \hfill
    % --- Passo 3: Riduzione F -> T (T sale a livello 2) ---
    \begin{tikzpicture}[baseline=0cm]
        % Albero T posizionato a y=40pt (Livello 2)
        \node at (0, 40pt) {T}
            child {node {F}
                child {node {\textbf{id}}}
            };
        \node [right=0.2cm] at (0,0) {$*$ \textbf{id}};
        \path (0,3);
    \end{tikzpicture}
    \hfill
    % --- Passo 4: Riduzione secondo id -> F ---
    \begin{tikzpicture}[baseline=0cm]
        % Primo albero (T) fermo a livello 2
        \node at (0, 40pt) {T}
            child {node {F}
                child {node {\textbf{id}}}
            };
        % Simbolo * in basso
        \node at (0.35, 0) {$*$};
        % Secondo albero (F) a livello 1
        \node at (0.7, 20pt) {F}
            child {node {\textbf{id}}};
        \path (0,3);
    \end{tikzpicture}
    \hfill
    % --- Passo 5: Riduzione T * F -> T (Nuovo T sale a livello 3) ---
    \begin{tikzpicture}[baseline=0cm]
        % Albero T risultante a y=60pt (Livello 3)
        \node at (0.35, 60pt) {T}
            child {node {T}
                child {node {F}
                    child {node {\textbf{id}}}
                }
            }
            child {node {$*$}}
            child {node {F}
                child {node {\textbf{id}}}
            };
        \path (0,3);
    \end{tikzpicture}
    \hfill
    % --- Passo 6: Riduzione T -> E (E sale a livello 4) ---
    \begin{tikzpicture}[baseline=0cm]
        % Albero E finale a y=80pt (Livello 4)
        \node at (0.35, 80pt) {E}
            child {node {T}
                child {node {T}
                    child {node {F}
                        child {node {\textbf{id}}}
                    }
                }
                child {node {$*$}}
                child {node {F}
                    child {node {\textbf{id}}}
                }
            };
        \path (0,3);
    \end{tikzpicture}
\end{figure}

\vspace{2 cm}
Sequenze con somma:

\begin{figure}[h]
    \centering
    % Configurazione stile alberi
    \tikzset{
        level distance=20pt,      
        sibling distance=5pt,     
        every node/.style={inner sep=1pt, font=\small},
        edge from parent/.style={draw, thin}
    }

    % --- Passo 1: Stringa iniziale ---
    \begin{tikzpicture}[baseline=0cm]
        \node at (0,0) {\textbf{id}};
        \node at (0.35,0) {$+$};
        \node at (0.7,0) {\textbf{id}};
        \path (0,3); % Spazio verticale per allineamento
    \end{tikzpicture}
    \hfill
    % --- Passo 2: Primo id -> F ---
    \begin{tikzpicture}[baseline=0cm]
        \node at (0, 20pt) {F}
            child {node {\textbf{id}}};
        \node at (0.35,0) {$+$};
        \node at (0.7,0) {\textbf{id}};
        \path (0,3);
    \end{tikzpicture}
    \hfill
    % --- Passo 3: F -> T ---
    \begin{tikzpicture}[baseline=0cm]
        \node at (0, 40pt) {T}
            child {node {F}
                child {node {\textbf{id}}}
            };
        \node at (0.35,0) {$+$};
        \node at (0.7,0) {\textbf{id}};
        \path (0,3);
    \end{tikzpicture}
    \hfill
    % --- Passo 4: T -> E (Per preparare la somma E + T) ---
    \begin{tikzpicture}[baseline=0cm]
        \node at (0, 60pt) {E}
            child {node {T}
                child {node {F}
                    child {node {\textbf{id}}}
                }
            };
        \node at (0.35,0) {$+$};
        \node at (0.7,0) {\textbf{id}};
        \path (0,3);
    \end{tikzpicture}
    
    \vspace{1cm} % Andiamo a capo per la seconda riga di passaggi
    
    % --- Passo 5: Secondo id -> F ---
    \begin{tikzpicture}[baseline=0cm]
        % Albero E a sinistra
        \node at (0, 60pt) {E}
            child {node {T}
                child {node {F}
                    child {node {\textbf{id}}}
                }
            };
        \node at (0.35,0) {$+$};
        % Albero F a destra
        \node at (0.7, 20pt) {F}
            child {node {\textbf{id}}};
        \path (0,3);
    \end{tikzpicture}
    \hfill
    % --- Passo 6: F -> T ---
    \begin{tikzpicture}[baseline=0cm]
        % Albero E a sinistra
        \node at (0, 60pt) {E}
            child {node {T}
                child {node {F}
                    child {node {\textbf{id}}}
                }
            };
        \node at (0.35,0) {$+$};
        % Albero T a destra
        \node at (0.7, 40pt) {T}
            child {node {F}
                child {node {\textbf{id}}}
            };
        \path (0,3);
    \end{tikzpicture}
    \hfill
    % --- Passo 7: Riduzione E + T -> E ---
    \begin{tikzpicture}[baseline=0cm]
        \node at (0.35, 80pt) {E}
            child {node {E}
                child {node {T}
                    child {node {F}
                        child {node {\textbf{id}}}
                    }
                }
            }
            child {node {$+$}}
            child {node {T}
                child {node {F}
                    child {node {\textbf{id}}}
                }
            };
        \path (0,3);
    \end{tikzpicture}
\end{figure}

Gli analizzatori bottom-up partono da una stringa \( w \) e procedono a ritroso, effettuando una progressiva riduzione fino ad ottenere il simbolo distinto \( S \).

\vspace{1em}
I parser bottom-up si basano sul meccanismo di riduzione che consiste nel sostituire la parte destra di una regola con la parte sinistra.

Una riduzione è l’inverso di un passo di derivazione, dove si espande una variabile con la parte destra di una regola, quindi un parser bottom-up ricostruisce in modo inverso una derivazione.

\vspace{1em}
Per gli esempi precedenti, considerando le radici dei sottoalberi, si hanno le sequenze di stringhe:

\[
id * id, \quad F * id, \quad T * id, \quad T * F, \quad T, \quad E,
\]

\[
id + id, \quad F + id, \quad T + id, \quad E + id, \quad E + F, \quad E + T, \quad E,
\]

corrispondenti alle derivazioni destre:

\[
E \Rightarrow T \Rightarrow T * F \Rightarrow T * id \Rightarrow F * id \Rightarrow id * id,
\]

\[
E \Rightarrow E + T \Rightarrow E + F \Rightarrow E + id \Rightarrow T + id \Rightarrow F + id \Rightarrow id + id.
\]

\vspace{1em}
Ad ogni passo, i parser bottom-up effettuano una riduzione oppure scandiscono un simbolo in ingresso. Per questo sono detti parser \textit{shift-reduce}, \textit{impila-riduci} o \textit{sposta-riduci}.

Le decisioni fondamentali sono se effettuare una riduzione e quale regola utilizzare.

Il parsing bottom-up scandisce una stringa da sinistra a destra e costruisce una derivazione destra.

Una \textbf{maniglia} (handle) è una sottostringa corrispondente alla parte destra di una regola la cui riduzione rappresenta un passo nella derivazione destra a ritroso.

\vspace{1em}
\textbf{Esempi di handle e regole di riduzione:}
\[
\begin{array}{lll}
\text{Forma di frase} & \text{Handle} & \text{Regola di riduzione} \\
\hline
id * id & id & F \to id \\
F * id & F & T \to F \\
T * id & id & F \to id \\
T * F & T * F & T \to T * F \\
T & T & E \to T \\
id + id & id & F \to id \\
F + id & F & T \to F \\
T + id & T & E \to T \\
E + id & id & F \to id \\
E + F & F & T \to F \\
E + T & E + T & E \to E + T \\
E & - & - \\
\end{array}
\]

\vspace{1em}
Nel primo esempio, nella stringa \( T * id \), \( T \) non viene ridotta anche se è parte destra della regola \( E \to T \).

Nel secondo esempio, nella stringa \( T + id \), \( T \) viene invece ridotta con la regola \( E \to T \).

Una sottostringa sinistra corrispondente alla parte destra di una regola non è necessariamente un handle.

Formalmente, se \( S \Rightarrow^{*} \alpha A w \Rightarrow \alpha \beta w \), la produzione \( A \to \beta \) nella posizione subito dopo \( \alpha \) è un handle di \( \alpha \beta w \).

\vspace{1em}
Un handle per una forma di frase destra \( \gamma \) è costituito dalla produzione \( A \to \beta \) e da una posizione in \( \gamma \) in cui si trova la stringa \( \beta \), tale che sostituendola con \( A \) si ottiene la forma di frase precedente in una derivazione destra di \( \gamma \).

In genere, con handle si intende la parte destra \( \beta \) della produzione.

La stringa \( w \) a destra dell’handle deve contenere solo simboli terminali: \( S \Rightarrow^{*} \alpha A w \Rightarrow \alpha \beta w \).

Se la grammatica è non ambigua, per ogni forma di frase destra c’è un unico handle.

Una derivazione destra può essere ottenuta a ritroso con un processo detto di \textit{potatura} o \textit{handle pruning}.

\newpage
\section*{Parsing Shift-Reduce}

Nel parsing \textit{impila-riduci} si usa uno stack che può contenere variabili e terminali, oltre al marcatore di fine stringa \( \$ \), e un buffer di ingresso che contiene la parte di input ancora da analizzare.

Un handle, subito prima di essere individuato, si trova sempre in cima allo stack.

Il simbolo \( \$ \) viene utilizzato come marcatore di fine stringa e per indicare il fondo dello stack. Inizialmente lo stack contiene \( \$ \) e in ingresso si ha \( w\$ \).

La stringa in ingresso viene scandita da sinistra a destra. Il parser inserisce nello stack (shift) zero o più simboli finché in cima non si trova un handle \( \beta \). A questo punto viene effettuata una riduzione (reduce) sostituendo \( \beta \) con la parte sinistra della regola opportuna.

Il parser ripete questo procedimento fino a rilevare un errore oppure quando lo stack contiene \( \$ S \) e in ingresso è rimasto solo \( \$ \).

\vspace{1em}
Ad ogni passo sono possibili 4 azioni:
\begin{enumerate}
    \item \textbf{Shift}: inserisce il prossimo simbolo in ingresso in cima allo stack.
    \item \textbf{Reduce}: il simbolo più a destra della stringa da ridurre si trova in cima allo stack. Si effettua una riduzione sostituendo la parte destra della regola con la parte sinistra.
    \item \textbf{Accept}: indica il corretto completamento dell’analisi.
    \item \textbf{Error}: si è verificata una situazione di errore.
\end{enumerate}

\vspace{1em}
Nell’esempio, lo stack viene rappresentato con l’elemento di testa a destra (così il contenuto dello stack e l’input da scandire, letti di seguito, corrispondono alle forme di frase destra).

\bigskip
\noindent\textbf{Esempio di parsing shift-reduce:}

\begin{table}[h!]
    \centering
    \begin{tabular}{|c|c|c|}
    \hline
    \textbf{Stack} & \textbf{Input} & \textbf{Azione} \\
    \hline
    \$ & \texttt{aaaaccbb\$} & \textit{shift} \\
    \$ a & \texttt{aaaccbb\$} & \textit{shift} \\
    \$ aa & \texttt{aaccbb\$} & \textit{shift} \\
    \$ aaa & \texttt{accbb\$} & \textit{shift} \\
    \$ aaaa & \texttt{ccbb\$} & \textit{shift} \\
    \$ \textcolor{teal}{aaaac} & \texttt{cbb\$} & \textit{reduce $A \to ac$} \\
    \$ aaaA & \texttt{cbb\$} & \textit{shift} \\
    \$ \textcolor{teal}{aaaAc} & \texttt{bb\$} & \textit{reduce $A \to aAc$} \\
    \$ aaA & \texttt{bb\$} & \textit{shift} \\
    \$ \textcolor{teal}{aaAb} & \texttt{b\$} & \textit{reduce $S \to aAb$} \\
    \$ aS & \texttt{b\$} & \textit{shift} \\
    \$ \textcolor{teal}{aSb} & \texttt{\$} & \textit{reduce $S \to aSb$} \\
    \$ S & \texttt{\$} & \textit{accept} \\
    \hline
    \end{tabular}
    \end{table}

Grammatica: 
\[
S \to aSb \mid aAb, \quad A \to aAc \mid ac
\]

\[
S \Rightarrow aSb \Rightarrow aaAbb \Rightarrow aaaAcbb \Rightarrow aaaaccbb
\]

\vspace{10 cm}
\subsubsection*{Esempio parsing shift-reduce con prodotto}

\begin{table}[h!]
    \centering
    \renewcommand{\arraystretch}{1.2}
    \begin{tabular}{|c|c|c|}
    \hline
    \textbf{Stack} & \textbf{Input} & \textbf{Azione} \\
    \hline
    \$ & \textbf{id * id \$} & \textit{shift} \\
    \textcolor{teal}{\$ id} & \textbf{* id \$} & \textit{reduce $F \rightarrow \textbf{id}$} \\
    \textcolor{teal}{\$ \textit{F}} & \textbf{* id \$} & \textit{reduce $T \rightarrow F$} \\
    \textcolor{teal}{\$ \textit{T}} & \textbf{* id \$} & \textit{shift} \\
    \textcolor{teal}{\$ \textit{T} *} & \textbf{id \$} & \textit{shift} \\
    \textcolor{teal}{\$ \textit{T} * id} & \textbf{\$} & \textit{reduce $F \rightarrow \textbf{id}$} \\
    \textcolor{teal}{\$ \textit{T} * \textit{F}} & \textbf{\$} & \textit{reduce $T \rightarrow T * F$} \\
    \textcolor{teal}{\$ \textit{T}} & \textbf{\$} & \textit{reduce $E \rightarrow T$} \\
    \textcolor{teal}{\$ \textit{E}} & \textbf{\$} & \textit{accept} \\
    \hline
    \end{tabular}
    \end{table}

\vspace{1em}

\subsubsection*{Esempio di parsing shift-reduce con somma:}

\begin{table}[h!]
    \centering
    \renewcommand{\arraystretch}{1.2}
    \begin{tabular}{|c|c|c|}
    \hline
    \textbf{Stack} & \textbf{Input} & \textbf{Azione} \\
    \hline
    \$ & \textbf{id + id \$} & \textit{shift} \\
    \textcolor{teal}{\$ id} & \textbf{+ id \$} & \textit{reduce $F \rightarrow \textbf{id}$} \\
    \textcolor{teal}{\$ F} & \textbf{+ id \$} & \textit{reduce $T \rightarrow F$} \\
    \textcolor{teal}{\$ T} & \textbf{+ id \$} & \textit{reduce $E \rightarrow T$} \\
    \textcolor{teal}{\$ E} & \textbf{+ id \$} & \textit{shift} \\
    \textcolor{teal}{\$ E +} & \textbf{id \$} & \textit{shift} \\
    \textcolor{teal}{\$ E + id} & \textbf{\$} & \textit{reduce $F \rightarrow \textbf{id}$} \\
    \textcolor{teal}{\$ E + F} & \textbf{\$} & \textit{reduce $T \rightarrow F$} \\
    \textcolor{teal}{\$ E + T} & \textbf{\$} & \textit{reduce $E \rightarrow E + T$} \\
    \textcolor{teal}{\$ E} & \textbf{\$} & \textit{accept} \\
    \hline
    \end{tabular}
    \end{table}
    
\vspace{1em}
L’uso dello stack è giustificato dal fatto che l’handle, a un certo punto, si trova sempre in cima allo stack, mai al suo interno.

Si può argomentare osservando due possibili casi per due passi successivi in una derivazione destra.

\vspace{1em}
\textbf{Caso 1:}
\begin{itemize}
    \item \( A \) è sostituito da \( \beta B y \)
    \item \( B \) (variabile più a destra in \( \beta B y \)) è sostituito da \( \gamma \)
\end{itemize}
La derivazione:

\[
S \Rightarrow^* \alpha A z \Rightarrow \alpha \beta B y z \Rightarrow \alpha \beta \gamma y z
\]

Letta a rovescio, quando il parser ha raggiunto la configurazione:

\[
\text{Stack: } \$ \alpha \beta \gamma \quad \text{Input: } y z \$
\]

Il parser riduce l’handle \( \gamma \) a \( B \) e ottiene:

\[
\$ \alpha \beta B \quad y z \$
\]

Poi può inserire \( y \) nello stack con una sequenza di zero o più shift, raggiungendo la configurazione:

\[
\$ \alpha \beta B y \quad z \$
\]

Con l’handle \( \beta B y \) in cima allo stack, che viene poi ridotto ad \( A \).

\vspace{1em}
\textbf{Caso 2:}
\begin{itemize}
    \item \( A \) è espanso sostituito da una stringa \( y \) di soli terminali
    \item \( B \) (variabile più a destra) si trova a sinistra di \( y \)
\end{itemize}
La derivazione:

\[
S \Rightarrow^* \alpha B x A z \Rightarrow \alpha B x y z \Rightarrow \alpha \gamma y z
\]

Nella configurazione:

\[
\$ \alpha \gamma \quad x y z \$
\]

L’handle \( \gamma \) è in cima allo stack. Dopo la riduzione di \( \gamma \) a \( B \), il parser inserisce \( x y \) nello stack, ottenendo in cima l’handle successivo da ridurre ad \( A \):

\[
\$ \alpha B x y \quad z \$
\]

\vspace{1em}
In entrambi i casi, dopo una riduzione, il parser deve fare zero o più shift per avere in cima allo stack l’handle successivo.

Il parser non deve mai analizzare l’interno dello stack per trovare l’handle.

\vspace{1 cm}
\textbf{Problema fondamentale nel parsing shift-reduce:} Quando in cima allo stack si ha una parte destra di una regola, come si fa a sapere se è un handle e quindi si deve fare la riduzione, oppure se è necessario ancora uno shift?

In cima allo stack potrebbero esserci parti destre di due regole diverse: quale si sceglie?

\vspace{10 cm}
\textbf{Esempio complesso di analisi:}


\begin{table}[h!]
    \centering
    \renewcommand{\arraystretch}{1.2}
    \begin{tabular}{|c|c|c|}
    \hline
    \textbf{Stack} & \textbf{Input} & \textbf{Azione} \\
    \hline
    \$ & \textbf{id + id * id \$} & \textit{shift} \\
    \textcolor{teal}{\$\;\textit{id}} & \textbf{+ id * id \$} & \textit{reduce $F \rightarrow $\textbf{id}} \\
    \textcolor{teal}{\$\;\textit{F}} & \textbf{+ id * id \$} & \textit{reduce $T \rightarrow F$} \\
    \textcolor{teal}{\$\;\textit{T}} & \textbf{+ id * id \$} & \textit{reduce $E \rightarrow T$} \\
    \textcolor{teal}{\$\;\textit{E}} & \textbf{+ id * id \$} & \textit{shift} \\
    \textcolor{teal}{\$\;\textit{E}+} & \textbf{id * id \$} & \textit{shift} \\
    \textcolor{teal}{\$\;\textit{E}+\,\textit{id}} & \textbf{* id \$} & \textit{reduce $F \rightarrow $\textbf{id}} \\
    \textcolor{teal}{\$\;\textit{E}+\,\textit{F}} & \textbf{* id \$} & \textit{reduce $T \rightarrow F$} \\
    \textcolor{teal}{\$\;\textit{E}+\,\textit{T}} & \textbf{* id \$} & \textit{shift} \\
    \textcolor{teal}{\$\;\textit{E}+\,\textit{T}*} & \textbf{id \$} & \textit{shift} \\
    \textcolor{teal}{\$\;\textit{E}+\,\textit{T}*\,\textit{id}} & \textbf{\$} & \textit{reduce $F \rightarrow $\textbf{id}} \\
    \textcolor{teal}{\$\;\textit{E}+\,\textit{T}*\,\textit{F}} & \textbf{\$} & \textit{reduce $T \rightarrow T * F$} \\
    \textcolor{teal}{\$\;\textit{E}+\,\textit{T}} & \textbf{\$} & \textit{reduce $E \rightarrow E + T$} \\
    \textcolor{teal}{\$\;\textit{E}} & \textbf{\$} & \textit{accept} \\
    \hline
    \end{tabular}
    \end{table}

\begin{table}[h!]
\centering
\renewcommand{\arraystretch}{1.2}
\begin{tabular}{|c|c|c|}
\hline
\textbf{Stack} & \textbf{Input} & \textbf{Azione} \\
\hline
\$ & \textbf{id + id + id \$} & \textit{shift} \\
\textcolor{teal}{\$\;\textit{id}} & \textbf{+ id + id \$} & \textit{reduce $F \rightarrow $\textbf{id}} \\
\textcolor{teal}{\$\;\textit{F}} & \textbf{+ id + id \$} & \textit{reduce $T \rightarrow F$} \\
\textcolor{teal}{\$\;\textit{T}} & \textbf{+ id + id \$} & \textit{reduce $E \rightarrow T$} \\
\textcolor{teal}{\$\;\textit{E}} & \textbf{+ id + id \$} & \textit{shift} \\
\textcolor{teal}{\$\;\textit{E}+} & \textbf{id + id \$} & \textit{shift} \\
\textcolor{teal}{\$\;\textit{E}+\,\textit{id}} & \textbf{+ id \$} & \textit{reduce $F \rightarrow $\textbf{id}} \\
\textcolor{teal}{\$\;\textit{E}+\,\textit{F}} & \textbf{+ id \$} & \textit{reduce $T \rightarrow F$} \\
\textcolor{teal}{\$\;\textit{E}+\,\textit{T}} & \textbf{+ id \$} & \textit{reduce $E \rightarrow E + T$} \\
\textcolor{teal}{\$\;\textit{E}} & \textbf{+ id \$} & \textit{shift} \\
\textcolor{teal}{\$\;\textit{E}+} & \textbf{id \$} & \textit{shift} \\
\textcolor{teal}{\$\;\textit{E}+\,\textit{id}} & \textbf{\$} & \textit{reduce $F \rightarrow $\textbf{id}} \\
\textcolor{teal}{\$\;\textit{E}+\,\textit{F}} & \textbf{\$} & \textit{reduce $T \rightarrow F$} \\
\textcolor{teal}{\$\;\textit{E}+\,\textit{T}} & \textbf{\$} & \textit{reduce $E \rightarrow E + T$} \\
\textcolor{teal}{\$\;\textit{E}} & \textbf{\$} & \textit{accept} \\
\hline
\end{tabular}
\end{table}



\section*{Parsing LR: concetti di base}
Il parsing usato comunemente per risolvere il problema è \textbf{LR(k)}:
\begin{itemize}
\item L (\textit{Left}): l’input viene letto da sinistra a destra.
\item R (\textit{Rightmost}): si costruisce una derivazione destra.
\item k: il parser può guardare i prossimi k simboli d’ingresso (di solito $k=1$).
\end{itemize}

\section*{Parsing LR: caratteristiche}
\begin{itemize}
    \item Si basa sulle tabelle di parsing.
    \item Una grammatica per la quale si può costruire la tabella si dice \textbf{LR}.
    \item Il parsing LR è adatto ai linguaggi di programmazione (grammatiche context-free).
    \item È il metodo shift-reduce più generale senza backtracking.
    \item Identifica errori sintattici appena possibile.
    \item Riconosce una classe più ampia delle grammatiche rispetto a LL.
\end{itemize}

\section*{Item LR(0) e stati}
Un parser LR prende le decisioni sposta/riduci memorizzando lo stato attuale.  
Ogni stato rappresenta un insieme di \emph{item}. Un \textbf{item LR(0)} di una grammatica $G$ è una produzione con un punto ($\cdot$) in una certa posizione della parte destra.  
Ad esempio, dato $A \to XYZ$ ci sono 4 item:
\begin{align*}
A &\to \cdot XYZ\\
A &\to X\cdot YZ\\
A &\to XY\cdot Z\\
A &\to XYZ\cdot
\end{align*}

La produzione $A \to \varepsilon$ genera solo $A \to \cdot$.

Ogni item indica che prefisso della produzione è già stato letto.

\section*{Significato degli item}
\begin{itemize}
    \item $A \to \cdot XYZ$: ci si aspetta in ingresso una stringa derivabile da $XYZ$.
    \item $A \to X\cdot YZ$: si sono riconosciuti simboli derivabili da $X$, ora ci si aspetta $YZ$.
    \item $A \to XYZ\cdot$: si è appena riconosciuta una stringa derivabile da $XYZ$, si può ridurre usando questa regola.
\end{itemize}

La \textbf{collezione canonica} LR(0) è un insieme di insiemi di item LR(0) e permette di costruire un automa a stati finiti deterministico (automa LR(0)), che serve per guidare le decisioni nel parsing.

\section*{Chiusura degli insiemi di item}
Per costruire la collezione canonica, si parte dalla grammatica aumentata $G'$, ottenuta introducendo $S' \to S$ per l’accettazione.

Sia $I$ un insieme di item di $G$. Definiamo la funzione di chiusura:
\begin{enumerate}
    \item Inizialmente $CLOSURE(I)$ contiene tutti gli item di $I$.
    \item Se $A \to \alpha \cdot B \beta$ è in $CLOSURE(I)$ e $B \to \gamma$ è una regola di $G$, si aggiunge $B \to \cdot \gamma$ (se non già presente). Si ripete finché non si aggiungono più item.
\end{enumerate}

\noindent\textbf{Esempio:}  
Grammatica aumentata:
\begin{equation*}
\begin{array}{l}
E' \to E \\
E \to E + T \mid T \\
T \to T * F \mid F \\
F \to (E) \mid id
\end{array}
\end{equation*}

Se $I = \{ [E' \to \cdot E] \}$, la $CLOSURE(I)$ conterrà:  
$E \to \cdot E + T$, $E \to \cdot T$, $T \to \cdot T * F$, $T \to \cdot F$, $F \to \cdot (E)$, $F \to \cdot id$.

Questa operazione di chiusura definisce il contenuto esatto del primo stato ($I_0$) dell'Automa LR(0) per la grammatica delle espressioni:

\[
\begin{array}{rcl}
E & \to & E + T \mid T \\
T & \to & T * F \mid F \\
F & \to & ( E ) \mid \mathbf{id}
\end{array}
\]
Partendo da questo stato iniziale, l'algoritmo calcola le transizioni (tramite la funzione \texttt{GOTO}) spostando il punto in avanti per ogni simbolo grammaticale. Ogni spostamento può generare un nuovo stato o portare a uno già esistente. 

Ripetendo questo processo iterativamente per tutti gli stati scoperti, si ottiene la \textbf{Collezione Canonica LR(0)}, visualizzata nel seguente Automa a Stati Finiti (DFA):

\vspace{1 cm}
\begin{figure}[H]
    \centering
    \begin{tikzpicture}[
        node distance=1.2cm and 1.5cm,
        box/.style={
            rectangle, 
            draw, 
            align=left, 
            font=\scriptsize,
            minimum width=1.8cm,
            inner sep=3pt,
            fill=white
        },
        arrow/.style={
            -Latex, 
            thick,
            rounded corners=3pt
        },
        label_edge/.style={
            pos=0.4,
            fill=white, 
            inner sep=1pt,
            font=\footnotesize\bfseries
        }
    ]

    % --- COLONNA PRINCIPALE (Sinistra) ---
    % I0: Stato Iniziale
    \node[box] (I0) {
        \textbf{I$_0$}\\
        $E' \to \cdot E$\\
        \color{gray}$E \to \cdot E + T$\\
        \color{gray}$E \to \cdot T$\\
        \color{gray}$T \to \cdot T * F$\\
        \color{gray}$T \to \cdot F$\\
        \color{gray}$F \to \cdot ( E )$\\
        \color{gray}$F \to \cdot \mathbf{id}$
    };

    % I2: Dopo aver letto T
    \node[box, below=1.8cm of I0] (I2) {
        \textbf{I$_2$}\\
        $E \to T \cdot$\\
        $T \to T \cdot * F$
    };

    % I4: Dopo aver letto '(' (Nodo centrale per ricorsioni)
    \node[box, below=1.8cm of I2] (I4) {
        \textbf{I$_4$}\\
        $F \to ( \cdot E )$\\
        \color{gray}$E \to \cdot E + T$\\
        \color{gray}... (closure)
    };

    % I3: Dopo aver letto F
    \node[box, below=1.8cm of I4] (I3) {
        \textbf{I$_3$}\\
        $T \to F \cdot$
    };

    % I5: Dopo aver letto id (Nodo centrale)
    \node[box] (I5) at ($(I2)!0.5!(I4)$) {
        \textbf{I$_5$}\\
        $F \to \mathbf{id} \cdot$
    };

    % --- COLONNA SUPERIORE (Dopo E) ---
    \node[box, right=2.5cm of I0] (I1) {
        \textbf{I$_1$}\\
        $E' \to E \cdot$\\
        $E \to E \cdot + T$
    };
    \node[below=0.2cm of I1, font=\bfseries] (acc) {accept};

    % --- COLONNA DESTRA (Dopo Operatori) ---
    
    % I6: Dopo E +
    \node[box, right=2.5cm of I1] (I6) {
        \textbf{I$_6$}\\
        $E \to E + \cdot T$\\
        \color{gray}$T \to \cdot T * F$\\
        \color{gray}$T \to \cdot F$\\
        \color{gray}...
    };

    % I7: Dopo T *
    \node[box, right=2.5cm of I2] (I7) {
        \textbf{I$_7$}\\
        $T \to T * \cdot F$\\
        \color{gray}$F \to \cdot ( E )$\\
        \color{gray}$F \to \cdot \mathbf{id}$
    };

    % I8: Dopo ( E
    \node[box, right=2.5cm of I4] (I8) {
        \textbf{I$_8$}\\
        $F \to ( E \cdot )$\\
        $E \to E \cdot + T$
    };

    % --- COLONNA ESTREMA DESTRA (Finali) ---
    
    % I9: Dopo E + T
    \node[box, right=2cm of I6] (I9) {
        \textbf{I$_9$}\\
        $E \to E + T \cdot$\\
        $T \to T \cdot * F$
    };

    % I10: Dopo T * F
    \node[box, right=2cm of I7] (I10) {
        \textbf{I$_{10}$}\\
        $T \to T * F \cdot$
    };

    % I11: Dopo ( E )
    \node[box, right=2cm of I8] (I11) {
        \textbf{I$_{11}$}\\
        $F \to ( E ) \cdot$
    };


    % --- COLLEGAMENTI ---

    % Da I0
    \draw[arrow] (I0) -- node[above] {E} (I1);
    \draw[arrow] (I1) -- (acc);
    \draw[arrow] (I0) -- node[left] {T} (I2);
    % Collegamenti lunghi da I0 a I4 e I5 usando percorsi a 'L'
    \draw[arrow] (I0.west) -- ++(-0.5,0) |- node[label_edge, near start] {(} (I4.west);
    \draw[arrow] (I0) |- node[label_edge, near end] {id} (I5); 
    % Nota: collegamento I0->I3 per F
    \draw[arrow] (I0.west) -- ++(-0.8,0) |- node[label_edge, near start] {F} (I3.west);

    % Da I1
    \draw[arrow] (I1) -- node[above] {+} (I6);

    % Da I2
    \draw[arrow] (I2) -- node[above] {*} (I7);

    % Da I4 (Il nodo più complesso)
    \draw[arrow] (I4) -- node[above] {E} (I8);
    \draw[arrow] (I4) edge[loop left] node[left] {(} (I4);
    \draw[arrow] (I4) -- node[left] {id} (I5);
    \draw[arrow] (I4) -- node[left] {T} (I2); % Risale a I2
    \draw[arrow] (I4) -- node[left] {F} (I3); % Scende a I3

    % Da I6 (Simile a I0 ma spostato)
    \draw[arrow] (I6) -- node[above] {T} (I9);
    % I6 verso id e (
    \draw[arrow] (I6.south) |- node[label_edge, near start] {id} (I5.east);
    \draw[arrow] (I6.south west) to[out=-135, in=45] node[label_edge] {(} (I4.north east);
    % I6 verso F -> I3 (Lungo giro)
    \draw[arrow] (I6.east) -- ++(0.5,0) |- node[label_edge, near start] {F} (I3.east);

    % Da I7
    \draw[arrow] (I7) -- node[above] {F} (I10);
    \draw[arrow] (I7) |- node[label_edge, near start] {id} (I5.east);
    \draw[arrow] (I7.south west) to[out=-135, in=20] node[label_edge] {(} (I4.north east);

    % Da I8
    \draw[arrow] (I8) -- node[above] {)} (I11);
    \draw[arrow] (I8) -- node[left] {+} (I6); % Risale per la somma

    % Da I9
    \draw[arrow] (I9) |- node[label_edge, near start] {*} (I7);

    \end{tikzpicture}
\end{figure}

\vspace{20 cm}

\section*{Algoritmi Closure e GOTO}
\textbf{Chiusura (pseudocodice):}
\begin{verbatim}
SetOfItems CLOSURE(I) {
    J = I
    repeat
        for (ogni item A -> alpha . B beta in J)
            for (ogni regola B -> gamma in G)
                if (B -> . gamma non in J)
                    aggiungi B -> . gamma a J;
    until nessun nuovo item è aggiunto a J;
    return J;
}
\end{verbatim}

\textbf{Funzione GOTO:}
Sia $I$ un insieme di item e $X$ un simbolo della grammatica,
\[
\mathrm{GOTO}(I, X) = \mathrm{CLOSURE}(\{ [A \to \alpha X \cdot \beta] \mid [A \to \alpha \cdot X \beta] \in I \})
\]
Usata per definire le transizioni dell’automa LR(0).

\section*{Collezione canonica LR(0)}
Costruita tramite:
\begin{verbatim}
void items(G') {
    C = CLOSURE({[S' -> .S ]});
    repeat
        for ( ogni insieme di item I in C )
            for ( ogni simbolo X in G )
                if (GOTO(I, X) non è vuoto e non appartiene a C )
                    aggiungi GOTO(I, X) a C;
    until nessun nuovo insieme di item è aggiunto a C;
}
\end{verbatim}
Gli stati dell’automa sono gli insiemi di item. La funzione di transizione è data da GOTO.

\section*{Parsing LR e la pila degli stati}

Durante il parsing si usa una tabella con più colonne:
\begin{itemize}
    \item Stack degli stati
    \item Stack dei simboli grammaticali
    \item Input residuo
    \item Azione (Shift, Reduce)
\end{itemize}
Ad ogni passo, se dallo stato $j$ c’è transizione col prossimo simbolo in ingresso $a$, allora si sceglie di impilare $a$. Altrimenti si effettua una riduzione.

\textbf{Quando si applica una riduzione $A \to X_1 X_2 \ldots X_n$}, si rimuovono $n$ stati dalla pila e si usa la funzione di transizione sull’ultimo stato rimasto e su $A$.

\section*{Esempio tabella di parsing}
\begin{table}[H]
    \centering
    \renewcommand{\arraystretch}{1.25}
    \begin{tabular}{|c|c|c|c|}
    \hline
    \textbf{Stack} & \textbf{Simboli} & \textbf{Input} & \textbf{Azione} \\
    \hline
    0 & $\$ $ & $\mathbf{id * id\ \$}$ & \textit{shift 5} \\
    0 5 & $\$\mathbf{id}$ & $\mathbf{* id\ \$}$ & \textit{reduce $F \rightarrow \mathbf{id}$} \\
    0 3 & $\$F$ & $\mathbf{* id\ \$}$ & \textit{reduce $T \rightarrow F$} \\
    0 2 & $\$T$ & $\mathbf{* id\ \$}$ & \textit{shift 7} \\
    0 2 7 & $\$T*$ & $\mathbf{id\ \$}$ & \textit{shift 5} \\
    0 2 7 5 & $\$T*\mathbf{id}$ & $\mathbf{\$}$ & \textit{reduce $F \rightarrow \mathbf{id}$} \\
    0 2 7 10 & $\$T*F$ & $\mathbf{\$}$ & \textit{reduce $T \rightarrow T * F$} \\
    0 2 & $\$T$ & $\mathbf{\$}$ & \textit{reduce $E \rightarrow T$} \\
    0 1 & $\$E$ & $\mathbf{\$}$ & \textit{accept} \\
    \hline
    \end{tabular}
    \end{table}
    
    L'esempio illustra le azioni del  parser shift-reduce durante l'analisi della stringa $\mathbf{id} * \mathbf{id}$, secondo l'automa LR(0), utilizzando uno stack per memorizzare gli stati dell'automa.

    Il processo si basa su due meccanismi principali:
    
    \begin{enumerate}
        \item \textbf{Lo Shift (Spostamento):}
        Quando lo stato in cima allo stack ha una transizione etichettata con il simbolo corrente in ingresso, il parser "sposta" il simbolo nello stack.
        \begin{itemize}
            \item \textit{Esempio:} All'inizio (linea 1), lo stack contiene lo stato iniziale \textbf{0}. Il simbolo in ingresso è $\mathbf{id}$. Poiché esiste una transizione $0 \xrightarrow{\mathbf{id}} 5$, il parser impila lo stato \textbf{5} (che corrisponde al simbolo $\mathbf{id}$).
        \end{itemize}
    
        \item \textbf{La Riduzione (Reduce):}
        Quando lo stato corrente non ha transizioni uscenti per il simbolo in ingresso, ma contiene un item completo (es. $F \to \mathbf{id} \cdot$), si effettua una riduzione.
        \begin{itemize}
            \item \textit{Esempio:} Alla linea 2, siamo nello stato \textbf{5} e l'input è $*$. Non ci sono transizioni uscenti, quindi si riduce usando $F \to \mathbf{id}$.
            \item \textbf{Meccanica della riduzione:}
            \begin{enumerate}
                \item Si rimuove dallo stack il corpo della produzione (in questo caso lo stato 5, corrispondente a $\mathbf{id}$).
                \item Lo stato che riemerge in cima è lo stato \textbf{0}.
                \item Si cerca la transizione dallo stato 0 sul simbolo testa della produzione ($F$). Poiché $0 \xrightarrow{F} 3$, si impila lo stato \textbf{3}.
            \end{enumerate}
        \end{itemize}
    \end{enumerate}
    
    Un comportamento analogo avviene successivamente (linea 5): partendo dallo stato \textbf{7} (che corrisponde al simbolo $*$), se arriva un $\mathbf{id}$ si fa shift verso lo stato 5. Successivamente, riducendo $F \to \mathbf{id}$, si rimuove il 5, riemerge il 7, e poiché $7 \xrightarrow{F} 10$, si impila lo stato \textbf{10}.
\section*{Nota}
Se nello stato 2, in cima alla pila, il prossimo simbolo è $*$, si effettua shift.
Se invece si facesse una riduzione, si rischierebbe di portare la macchina in errore.




\section{Traduzione guidata dalla sintassi}

Associando attributi ai simboli grammaticali, si definiscono valori tramite regole semantiche. La traduzione guidata dalla sintassi costruisce un albero di parsing per calcolare questi valori.

\subsection{Definizioni guidate dalla sintassi (SSD)}

È una grammatica context-free con l'aggiunta di attributi e regole semantiche. Gli attributi sono associati ai simboli della grammatica, le regole semantiche sono associate alle produzioni.

\textbf{Osservazione:} Dato un simbolo della grammatica \(X\) e uno dei suoi attributi \(a\), si indica \(X.a\) (il valore di a per uno specifico nodo dell'albero di parsing con etichetta X).

Una SSD che contiene solo attributi sintetizzati è detta \textbf{S-attribuita}, cioè dove ogni regola calcola un attributo associato alla variabile della parte sinistra della produzione mediante gli attributi associati ai simboli della parte destra. Una SSD senza effetti collaterali (qualcosa che modifica lo stato del sistema oltre a restituire un valore, tipo stampare a video) è detta \textbf{grammatica ad attributi}.

\subsection{Valutazione di una SSD ai nodi di un albero di parsing}

Un albero di parsing annotato mostra i valori degli attributi associati ai nodi.

Gli attributi ereditati sono utili quando la struttura dell'albero di parsing non corrisponde alla sintassi astratta del codice sorgente.

\subsubsection{Attributi sintetizzati}
Un attributo sintetizzato relativo ad un nodo \(N\) è definito unicamente in base ai figli di \(N\) e a \(N\) stesso.

\subsubsection{Attributi ereditati}
Un attributo ereditato relativo ad un nodo \(N\) è definito unicamente in base al padre di \(N\), a \(N\) stesso e ai fratelli di \(N\) (non dai figli).

\textbf{NB!} I terminali non possono avere attributi ereditati.

\subsection{Esempio: \(3 \times 5 \times 4\)}

\subsubsection{Prima grammatica}

\textbf{Produzioni e regole semantiche:}

\begin{enumerate}
    \item \(T \to T * F\) \quad \(T.val = T.val \times F.val\)
    \item \(T \to F\) \quad \(T.val = F.val\)
    \item \(F \to \text{num}\) \quad \(F.val = \text{num}.lexval\)
\end{enumerate}

\textbf{Albero di parsing annotato:}

\begin{center}
\begin{tikzpicture}[
  level distance=1.5cm,
  level 1/.style={sibling distance=3cm},
  level 2/.style={sibling distance=2cm},
  level 3/.style={sibling distance=1.5cm}
]
\node {\(T.val =\)}
  child {node {\(T.val =\)}
    child {node {\(T.val\)}
      child {node {\(F.val\)}
        child {node {3}}
      }
    }
    child {node {*}}
    child {node {\(F.val\)}
      child {node {5}}
    }
  }
  child {node {*}}
  child {node {\(F.val\)}
    child {node {4}}
  };
\end{tikzpicture}
\end{center}

\textbf{Ordinamento di valutazione:} 3, *, 5, *, 4

\subsubsection{Seconda grammatica (con attributi ereditati)}

\textbf{Produzioni e regole semantiche:}

\begin{enumerate}
    \item \(T \to FT'\) \quad \(T'.inh = F.val\) \quad \(T.val = T'.syn\)
    \item \(T' \to *FT'_1\) \quad \(T'_1.inh = T'.inh \times F.val\) \quad \(T'.syn = T'_1.syn\)
    \item \(T' \to \varepsilon\) \quad \(T'.syn = T'.inh\)
    \item \(F \to \text{num}\) \quad \(F.val = \text{num}.lexval\)
\end{enumerate}

\textbf{Albero di parsing annotato:}

\begin{center}
\begin{tikzpicture}[
  level distance=2.2cm,
  level 1/.style={sibling distance=7cm},
  level 2/.style={sibling distance=4.5cm},
  level 3/.style={sibling distance=3.5cm},
  level 4/.style={sibling distance=2.5cm},
  every node/.style={align=center, text width=4.5cm, font=\small}
]
\node {\(T.val = T'.syn(60)\)}
  child {node[text width=3cm] {\(F.val =\) \\ \(\text{num}.lexval(3)\)}
    child {node[text width=1cm] {3}}
  }
  child {node[text width=4cm] {\(T'.inh = F.val(3)\) \\ \(T'.syn = T'_1.syn(60)\)}
    child {node[text width=1cm] {*}}
    child {node[text width=3cm] {\(F.val =\) \\ \(\text{num}.lexval(5)\)}
      child {node[text width=1cm] {5}}
    }
    child {node[text width=4.5cm] {\(T'_1.inh =\) \\ \(T'.inh \times F.val(15)\) \\ \(T'_1.syn = T'_2.syn(60)\)}
      child {node[text width=1cm] {*}}
      child {node[text width=3cm] {\(F.val =\) \\ \(\text{num}.lexval(4)\)}
        child {node[text width=1cm] {4}}
      }
      child {node[text width=4.5cm] {\(T'_2.inh =\) \\ \(T'_1.inh \times F.val(60)\) \\ \(T'_2.syn = T'_2.inh(60)\)}
        child {node[text width=1.5cm] {\(\varepsilon\)}}
      }
    }
  };
\end{tikzpicture}
\end{center}

\textbf{Ordinamento di valutazione:} 3, 5, *, 4, \(\varepsilon\)

\textbf{Calcolo dei valori:}
\begin{itemize}
    \item \(F.val = 3\)
    \item \(T'.inh = 3\)
    \item \(F.val = 5\)
    \item \(T'_1.inh = 3 \times 5 = 15\)
    \item \(F.val = 4\)
    \item \(T'_2.inh = 15 \times 4 = 60\)
    \item \(T'_2.syn = 60\)
    \item \(T'_1.syn = 60\)
    \item \(T'.syn = 60\)
    \item \(T.val = 60\)
\end{itemize}



\subsection{Grafi delle dipendenze}

\subsubsection{Esempio: \(1 * 2\)}

\textbf{Produzioni e regole semantiche:}

\begin{center}
\begin{tabular}{|l|l|}
\hline
\textbf{Produzione} & \textbf{Regole semantiche} \\
\hline
1) \(T \to FT'\) & \(T'.inh = F.val\) \\
 & \(T.val = T'.syn\) \\
\hline
2) \(T' \to * F T'_1\) & \(T'_1.inh = T'.inh \times F.val\) \\
 & \(T'.syn = T'_1.syn\) \\
\hline
3) \(T' \to \varepsilon\) & \(T'.syn = T'.inh\) \\
\hline
4) \(F \to \text{digit}\) & \(F.val = \text{digit}.lexval\) \\
\hline
\end{tabular}
\end{center}

\textbf{Grafo delle dipendenze:}

\begin{center}
\begin{tikzpicture}[
  node distance=1.5cm and 2cm,
  every node/.style={font=\footnotesize},
  arrow/.style={->, thick, >=stealth, dashed}
]

% Nodi disposti secondo l'immagine
\node (T9val) at (2,5) {\(T\) 9 \(val\)};
\node (F3val) at (-2,3.5) {\(F\) 3 \(val\)};
\node (inh5) at (2,3.5) {\(inh\) 5 \(T'\)};
\node (syn8) at (5,3.5) {8 \(syn\) \(T'\)};
\node (digit1) at (-2,2) {digit 1 \(lexval\)};
\node (star) at (0,2) {*};
\node (F4val) at (2,1.5) {\(F\) 4 \(val\)};
\node (inh6) at (4.5,1.5) {\(inh\) 6 \(T'\)};
\node (syn7) at (6.5,1.5) {7 \(syn\) \(T'\)};
\node (digit2) at (2,0) {digit 2 \(lexval\)};
\node (epsilon) at (4.5,0) {\(\varepsilon\)};

% Frecce come nell'immagine
% digit 1 -> F 3 val
\draw[arrow] (digit1) -- (F3val);

% F 3 val -> inh 5 T'
\draw[arrow] (F3val) to[out=0, in=180] (inh5);

% digit 2 -> F 4 val
\draw[arrow] (digit2) -- (F4val);

% F 4 val -> inh 6 T'
\draw[arrow] (F4val) -- (inh6);

% inh 5 T' -> inh 6 T' (curva)
\draw[arrow] (inh5) to[out=-20, in=150] (inh6);

% inh 6 T' -> syn 7 T' (arco sopra)
\draw[arrow] (inh6) to[out=60, in=120] (syn7);

% syn 7 T' -> syn 8 T'
\draw[arrow] (syn7) to[out=150, in=0] (syn8);

% syn 8 T' -> T 9 val (curva a destra)
\draw[arrow] (syn8) to[out=90, in=0] (T9val);

\end{tikzpicture}
\end{center}


\subsection{Ordinamento topologico}

Gli ordinamenti validi sono costituiti da sequenze \(N_1, N_2, \ldots, N_k\), tale che se esiste un arco dal nodo \(N_i\) al nodo \(N_j\) nel grafo delle dipendenze, allora deve essere \(i < j\).

\subsection{Grammatiche S-attribuite}

Una definizione guidata dalla sintassi è S-attribuita se e solo se ogni suo attributo è sintetizzato.

\subsection{Grammatiche L-attribuite}

Una definizione è L-attribuita se, nel grafo delle dipendenze, gli archi vanno solo da sinistra a destra. In particolare:
\begin{itemize}
    \item Gli attributi ereditati possono dipendere solo da attributi (ereditati o sintetizzati) dei simboli alla loro \textbf{sinistra} o da attributi ereditati del padre.
    \item Questo vincolo è necessario per evitare cicli e permettere la valutazione in una sola passata.
\end{itemize}

\paragraph{Effetti Controllati}
Non sempre le pure regole semantiche bastano. Spesso si introducono \textbf{effetti controllati} (es. \texttt{print}, aggiornamento della Symbol Table).
\begin{itemize}
    \item Questi effetti sono permessi in modo che la traduzione risulti corretta per qualsiasi ordinamento di valutazione valido.
    \item I vincoli di ordine imposti dagli effetti laterali vengono trattati come \textbf{archi impliciti} nel grafo delle dipendenze.
\end{itemize}

\subsection{Alberi sintattici}
È utile trasformare una stringa in ingresso in un albero che ne rappresenta la struttura gerarchica (albero sintattico) per facilitare le fasi successive della compilazione.

\subsection{Classi di SDT implementabili durante il parsing}
Esistono due classi principali di SDT che consentono l'implementazione della traduzione direttamente durante l'analisi sintattica:

\begin{enumerate}
    \item La grammatica sottostante può essere riconosciuta da un parser \textbf{LR} e la SDD è \textbf{S-attribuita} (approccio \textit{bottom-up}).
    \item La grammatica sottostante può essere riconosciuta da un parser \textbf{LL} e la SDD è \textbf{L-attribuita} (approccio \textit{top-down}).
\end{enumerate}

\subsubsection{Marcatori e proprietà}
Per gestire azioni semantiche in posizioni arbitrarie, si possono introdurre dei marcatori (non-terminali che producono $\epsilon$).

\textbf{Proprietà:} Se la grammatica arricchita con i marcatori può essere trattata da un dato metodo di parsing, allora lo SDT corrispondente può essere implementato durante il parsing stesso.

\subsection{SDT postfissi}
Un SDT si dice \textbf{postfisso} se, in ogni produzione, le azioni semantiche sono posizionate esclusivamente \textbf{alla fine} del corpo della produzione.
\[ A \to X Y Z \{ \text{azione} \} \]
Questi schemi sono ideali per l'implementazione durante il parsing bottom-up.
\subsubsection{Schema di traduzione postfisso - Calcolatrice}

\textbf{Produzioni, Regole sintattiche e SDT:}

\begin{center}
\begin{tabular}{|l|l|l|}
\hline
\textbf{Produzioni} & \textbf{Regole sintattiche} & \textbf{SDT} \\
\hline
\(L \to E\) \textbf{n} & \{\texttt{print}(\(E.val\))\} & \{\texttt{print}(\(E.val\))\} \\
\hline
\(E \to E + T\) & \{\(E.val = E_1.val + T.val\)\} & \{\(E.val = E_1.val + T.val\)\} \\
\hline
\(E \to T\) & \{\(E.val = T.val\)\} & \{\(E.val = T.val\)\} \\
\hline
\(T \to T * F\) & \{\(T.val = T_1.val \times F.val\)\} & \{\(T.val = T_1.val \times F.val\)\} \\
\hline
\(T \to F\) & \{\(T.val = F.val\)\} & \{\(T.val = F.val\)\} \\
\hline
\(F \to (E)\) & \{\(F.val = E.val\)\} & \{\(F.val = E.val\)\} \\
\hline
\(F \to \textbf{digit}\) & \{\(F.val = \textbf{digit}.lexval\)\} & \{\(F.val = \textbf{digit}.lexval\)\} \\
\hline
\end{tabular}
\end{center}

\subsection{Stack del parser con attributi}

\textbf{Rappresentazione dello stack:}

\begin{center}
\begin{tabular}{|c|c|c|}
\hline
 & \(XYZ\) & \\
\hline
 & \(X.x\) \(Y.y\) \(Z.z\) & \\
\hline
\multicolumn{3}{c}{\(\uparrow\)} \\
\multicolumn{3}{c}{top} \\
\end{tabular}
\end{center}

Dopo una riduzione \(A \to XYZ\):

\begin{center}
\begin{tabular}{|c|}
\hline
\(A\) \\
\hline
\(A.a\) \\
\hline
\(\uparrow\) \\
top \\
\end{tabular}
\end{center}

Dove \(A.a = f(X.x, Y.y, Z.z)\)

\subsection{Implementazione con manipolazione dello stack}

\textbf{Azioni con manipolazione esplicita dello stack:}

\begin{itemize}
    \item \(L \to E\) \textbf{n}: \{\texttt{print}(\(stack[top-1].val\)); \(top = top - 1\)\}
    \item \(E \to E + T\): \{\(E.val = E.val + T.val\)\}
    \item \(T \to T * F\): \{\(stack[top-2].val = stack[top-2].val \times stack[top].val\); \(top = top - 2\)\}
    \item \(F \to (E)\): \{\(stack[top-2].val = stack[top-1].val\); \(top = top - 2\)\}
\end{itemize}

\subsection{Esempio di parsing: \(3 * (5 + 2)\) \textbf{n}}

\textbf{Sequenza di parsing con lo stack:}

\begin{center}
\begin{tabular}{|c|c|c|c|c|c|c|c|c|}
\hline
& \(F\) & \(E\) & \(T\) & \(E+T\) & \(T*F\) & (\(E\)) & & \\
\hline
& \(\uparrow\) & & & & & & & \\
\hline
3 & * & ( & 5 & + & 2 & ) & \textbf{n} & \texttt{print} \\
\hline
\end{tabular}
\end{center}

\textbf{Evoluzione dello stack:}

\begin{itemize}
    \item Dopo applicazione di \(F \to (E)\): \(top\) viene decrementato
    \item Dopo riduzione: i valori vengono calcolati e memorizzati nella posizione corretta dello stack
\end{itemize}

\textbf{Posizioni dello stack durante il parsing:}

\begin{center}
\begin{tabular}{c}
top-2 \\
\(\uparrow\) \\
top-1 \\
\(\uparrow\) \\
top \\
\end{tabular}
\end{center}





\subsection{Schemi di traduzione con azioni interne alle produzioni}

\subsubsection{Definizione e comportamento}

Un'azione può essere inserita in qualsiasi posizione nel corpo di una produzione. Essa sarà eseguita non appena tutti i simboli grammaticali alla sua sinistra saranno stati consumati.

Quindi, in una produzione del tipo \(B \to X \{a\} Y\), l'azione \(a\) è eseguita non appena abbiamo riconosciuto \(X\), se \(X\) è un terminale, oppure tutti i terminali derivati da \(X\), se quest'ultimo è un non-terminale.

\textbf{Comportamento nel parsing:}

\begin{itemize}
    \item Nel parsing \textbf{bottom-up}: si esegue l'azione \(a\) non appena l'occorrenza in esame del simbolo \(X\) appare sulla cima dello stack
    \item Nel parsing \textbf{top-down}: si esegue l'azione \(a\) immediatamente prima di tentare l'espansione di \(Y\), se \(Y\) è un non-terminale, oppure prima di cercare \(Y\) in ingresso, se \(Y\) è un terminale
\end{itemize}

\subsection{Eliminazione della ricorsione sinistra dagli SDT}

\subsubsection{Motivazione e principi}

Poiché nessuna grammatica che presenti ricorsione sinistra può essere analizzata mediante parsing top-down, diventa fondamentale eliminare tale tipo di ricorsione.

Quando una grammatica è L-attribuita e la si trasforma per eliminare la ricorsione sinistra, dobbiamo gestire correttamente le azioni semantiche.

\textbf{Assunzione:} Le azioni semantiche non calcolano valori di attributi ma effettuano azioni sulle stringhe di simboli terminali.

\textbf{Principio:} Nel processo di trasformazione della grammatica, le azioni semantiche sono trattate come ulteriori simboli terminali.

\subsubsection{Trasformazione standard}

Le trasformazioni della grammatica preservano l'ordine dei terminali nella stringa generata. La strategia consiste nel sostituire produzioni della forma:
\[
A \to A\alpha \mid \beta
\]
con nuove produzioni usando un nuovo non-terminale \(R\):
\[
A \to \beta R
\]
\[
R \to \alpha R \mid \varepsilon
\]

\subsubsection{Trasformazione per SDD S-attribuite}

Per una singola produzione ricorsiva, una singola produzione non-ricorsiva, ed un singolo attributo del non-terminale ricorsivo sinistro.

\textbf{Grammatica originale con ricorsione sinistra:}

\begin{itemize}
    \item \(A \to A\alpha\) con azione \(\{A.a = g(A_1.a, \alpha.y)\}\)
    \item \(A \to X\) con azione \(\{A.a = f(X.x)\}\)
\end{itemize}

\textbf{Grammatica trasformata senza ricorsione sinistra:}

\begin{itemize}
    \item \(A \to X\) \(\{R.i = f(X.x)\}\) \(R\) \(\{A.a = R.s\}\)
    \item \(R \to \alpha\) \(\{R_1.i = g(R.i, \alpha.y)\}\) \(R_1\) \(\{R.s = R_1.s\}\)
    \item \(R \to \varepsilon\) \(\{R.s = R.i\}\)
\end{itemize}

Dove \(R.i\) è l'attributo ereditato e \(R.s\) è l'attributo sintetizzato di \(R\).

\subsubsection{Esempio - Eliminazione ricorsione nelle espressioni}

Si considerino le seguenti produzioni relative a \(E\), prese da uno SDT per la traduzione di espressioni dalla forma infissa alla forma postfissa.

\textbf{SDT con ricorsione sinistra:}

\begin{center}
\begin{tabular}{|l|l|}
\hline
\textbf{Produzioni} & \textbf{Azioni semantiche} \\
\hline
\(E \to E_1 + T\) & \(\{\text{print}('+');\}\) \\
\hline
\(E \to T\) & \\
\hline
\end{tabular}
\end{center}

\noindent Se applichiamo la trasformazione standard al non-terminale \(E\), identifichiamo le parti $\alpha$ (la parte ricorsiva) e $\beta$ (la base) trattando l'azione semantica come un simbolo terminale:

\begin{itemize}
    \item Il \textbf{resto della produzione ricorsiva sinistra} ($\alpha$) è:
    \[ \alpha = + T \; \{ \text{print}('+'); \} \]
    \item Il \textbf{corpo dell'altra produzione} ($\beta$) è:
    \[ \beta = T \]
\end{itemize}

Introducendo il nuovo non-terminale \(R\) (che rappresenta il resto), si applica la regola \(E \to \beta R\) e \(R \to \alpha R \mid \varepsilon\), ottenendo:

\textbf{SDT senza ricorsione sinistra:}

\begin{center}
\begin{tabular}{|l|}
\hline
\textbf{Produzioni risultanti} \\
\hline
\(E \to T R\) \\
\hline
\(R \to + T\) \(\{\text{print}('+');\}\) \(R\) \\
\hline
\(R \to \varepsilon\) \\
\hline
\end{tabular}
\end{center}

\subsubsection{Esempio completo - Calcolatrice da tavolo}

\textbf{Con ricorsione sinistra:}

\begin{itemize}
    \item \(E \to E_1 + T\) \(\{E.v = E_1.v + T.v\}\)
    \item \(E \to T\) \(\{E.v = T.v\}\)
    \item \(T \to T_1 \times F\) \(\{T.v = T_1.v \times F.v\}\)
    \item \(T \to F\) \(\{T.v = F.v\}\)
    \item \(F \to (E)\) \(\{F.v = E.v\}\)
    \item \(F \to \textbf{digit}\) \(\{F.v = \textbf{digit}.\text{lexval}\}\)
\end{itemize}

\textbf{Senza ricorsione sinistra:}

\begin{itemize}
    \item \(E \to T\) \(\{R.i = T.v\}\) \(R\) \(\{E.v = R.s\}\)
    \item \(R \to + T\) \(\{R_1.i = R.i + T.v\}\) \(R_1\) \(\{R.s = R_1.s\}\)
    \item \(R \to \varepsilon\) \(\{R.s = R.i\}\)
    \item \(T \to F\) \(\{X.i = F.v\}\) \(X\) \(\{T.v = X.s\}\)
    \item \(X \to \times F\) \(\{X_1.i = X.i \times F.v\}\) \(X_1\) \(\{X.s = X_1.s\}\)
    \item \(X \to \varepsilon\) \(\{X.s = X.i\}\)
\end{itemize}

\subsection{Procedure generali per SDT}

\subsubsection{Procedura in tre passi}

\begin{enumerate}
    \item Si costruisce l'albero di parsing, ignorando le azioni semantiche
    \item Si esamina ogni nodo interno \(N\) con produzione associata. Si aggiungono nuovi nodi figli di \(N\) corrispondenti alle azioni, in modo che scorrendo tutti i figli di \(N\) da sinistra a destra si abbiano esattamente gli stessi simboli e le stesse azioni semantiche
    \item Si visita l'albero in preordine e non appena si incontra un nodo annotato con un'azione la si esegue
\end{enumerate}

\subsubsection{Esempio con albero di parsing}

Per l'espressione \(3 \times 5 + 4\), l'output in notazione prefissa è: \(+ (\times 3 \ 5) \ 4\)

Le azioni \(\{\text{print}('+');\}\) e \(\{\text{print}('*');\}\) vengono eseguite visitando l'albero in preordine, stampando gli operatori prima degli operandi.


L'albero rappresenta il parsing dell'espressione \(3 \times 5 + 4\) con le azioni di stampa in notazione prefissa.

\begin{center}
\begin{tikzpicture}[
  level distance=1.6cm,
  level 1/.style={sibling distance=5cm},
  level 2/.style={sibling distance=3.5cm},
  level 3/.style={sibling distance=2.8cm},
  level 4/.style={sibling distance=2cm},
  level 5/.style={sibling distance=1.5cm},
  every node/.style={align=center, font=\small},
  action/.style={fill=yellow!40, rectangle, rounded corners, inner sep=3pt, font=\footnotesize}
]

\node (L) {\(L\)}
  child {node (E1) {\(E\)}
    child {node (E2) {\(E\)}
      child {node (T1) {\(T\)}
        child {node (T2) {\(T\)}
          child {node (F1) {\(F\)}
            child {node (d1) {digit}
              child {node[action] (a3) {\{\text{print}(3);\}}}
            }
          }
        }
        child {node (star) {*}}
        child {node (F2) {\(F\)}
          child {node (d2) {digit}
            child {node[action] (a5) {\{\text{print}(5);\}}}
          }
        }
      }
      child {node (plus) {+}}
      child {node (T3) {\(T\)}
        child {node (F3) {\(F\)}
          child {node (d3) {digit}
            child {node[action] (a4) {\{\text{print}(4);\}}}
          }
        }
      }
    }
    child {node (n) {\textbf{n}}}
  };

% Azioni semantiche con linee tratteggiate
\node[action, left=2cm of E2] (print_plus) {\{\text{print}('+');\}};
\node[action, left=2cm of T2] (print_star) {\{\text{print}('*');\}};

% Linee tratteggiate che collegano le azioni ai nodi
\draw[dashed, gray, thick] (print_plus) -- (E2);
\draw[dashed, gray, thick] (print_star) -- (T1);

\end{tikzpicture}
\end{center}

\textbf{Output in notazione prefissa:} + * 3 5 4

\textbf{Ordine di esecuzione delle azioni:}
\begin{enumerate}
    \item print('+')
    \item print('*')
    \item print(3)
    \item print(5)
    \item print(4)
\end{enumerate}

%lezione 20 novembre
\subsection{Trasformazione da SDD L-attribuita a SDT}
Regole per trasformare SDD $\to$ SDT per parsing Top-Down:

\begin{enumerate}
    \item \textbf{Azioni per Attributi Ereditati:} Si aggiungono le azioni che calcolano gli attributi ereditati di un non-terminale $A$ \textbf{immediatamente prima} dell'occorrenza di $A$ nel corpo della produzione.
    \item \textbf{Azioni per Attributi Sintetizzati:} Si aggiungono le azioni che calcolano un attributo sintetizzato relativo alla testa della produzione \textbf{alla fine} del corpo di quella produzione.
\end{enumerate}

\section*{Esempio: Traduzione dello Statement \texttt{while}}
Consideriamo la produzione:
\[ S \to \mathbf{while} \ (C) \ S_1 \]

Useremo i seguenti attributi per generare il codice intermedio richiesto:

\begin{enumerate}
    \item L'attributo \textbf{ereditato} $S.next$ indica l'etichetta relativa all'inizio del codice che deve essere eseguito dopo che $S$ è terminato.
    
    \item L'attributo \textbf{sintetizzato} $S.code$ è la porzione di codice intermedio che implementa uno statement $S$ e termina con un salto a $S.next$ (se necessario, o cade nel flusso successivo).
    
    \item L'attributo \textbf{ereditato} $C.true$ indica l'etichetta relativa all'inizio del codice che deve essere eseguito se $C$ risulta vera.
    
    \item L'attributo \textbf{ereditato} $C.false$ indica l'etichetta relativa all'inizio del codice che deve essere eseguito se $C$ risulta falsa.
    
    \item L'attributo \textbf{sintetizzato} $C.code$ rappresenta la porzione di codice intermedio che implementa la condizione $C$ e che salta a $C.true$ oppure $C.false$ a seconda del valore dell'espressione $C$.
\end{enumerate}

\subsection*{Regole Semantiche (SDD)}
Per generare le nuove etichette e definire il flusso, si applicano le seguenti regole:

\begin{align*}
& L1 = \text{new}(); \quad \text{(Genera etichetta inizio ciclo)} \\
& L2 = \text{new}(); \quad \text{(Genera etichetta inizio corpo)} \\
& S_1.next = L1; \quad \text{(Dopo il corpo, torna a valutare la condizione)} \\
& C.false = S.next; \quad \text{(Se falso, esce dal ciclo)} \\
& C.true = L2; \quad \text{(Se vero, esegue il corpo)} \\
& S.code = \textbf{label} \ || \ L1 \ || \ C.code \ || \ \textbf{label} \ || \ L2 \ || \ S_1.code
\end{align*}

\subsection*{Logica del Flusso di Controllo}
Il funzionamento del codice generato segue questo schema logico (rappresentazione del diagramma):

\begin{figure}[h]
  \centering
  \begin{tikzpicture}[
      node distance=3.5cm, % Aumentata la distanza orizzontale
      box/.style={
          draw, 
          minimum width=1.5cm, 
          minimum height=1cm, 
          align=center, 
          font=\large\bfseries
      },
      arrow/.style={
          ->, 
          >=Stealth, 
          thick,
          rounded corners=5pt
      }
  ]

  % --- 1. DEFINIZIONE DEI NODI ---
  
  % Nodo C (Condizione)
  \node[box] (C) {C};
  
  % Nodo S1 (Statement) - Posizionato più lontano per leggibilità
  \node[box, right=of C] (S1) {S\textsubscript{1}};

  % --- 2. COORDINATE PER IL BOX ESTERNO (S) ---
  % Calcolo coordinate per fare il rettangolo arancione grande abbastanza
  \coordinate (TopLeft) at ($(C.north west) + (-2, 2.5)$);
  \coordinate (BottomRight) at ($(S1.south east) + (2, -2)$);
  \coordinate (ExitPoint) at ($(BottomRight) + (0, 0.5)$); % Punto di uscita fittizio a destra

  % Disegno il box S (sfondo)
  \draw[orange!80!yellow, thick] (TopLeft) rectangle (BottomRight);
  
  % Etichetta grande "S" (spostata in basso a sinistra per non disturbare)
  \node[font=\Huge\bfseries\sffamily, orange!40!yellow, anchor=south west] 
      at ($(TopLeft |- BottomRight) + (0.2, 0.2)$) {S};

  % --- 3. FRECCE E FLUSSO ---

  % Ingresso L
  \draw[arrow, black] ($(C.west) - (1.5, 0)$) -- node[above] {L\textsubscript{1}} (C.west);
  \draw[arrow, orange!80!yellow, thick] ($(TopLeft |- C) + (-1, 0.5)$) -- node[above, black] {L} ($(TopLeft |- C) + (0.5, 0.5)$);

  % Percorso TRUE (Centrale)
  \draw[arrow, cyan, thick] (C.east) -- 
      node[above, text=black, yshift=2pt] {TRUE} 
      node[below, text=black] {L\textsubscript{2}} 
      (S1.west);
  
  % Nota sopra il percorso True
  \node[font=\small\sffamily, align=center] at ($(C)!0.5!(S1) + (0, 0.8)$) {Eseguo 1\textdegree{} istruzione};

  % Percorso LOOP (Ritorno da S1 a C) - Alto
  \draw[arrow, cyan, thick] (S1.north) -- ++(0, 1) coordinate(midup)
      -- ($(C.north) + (0, 1)$) 
      node[midway, above, text width=5cm, align=center, black] {Devo rifare controllo\\condizione} 
      -- (C.north);

  % Percorso FALSE (Uscita) - Basso
  \draw[arrow, green!60!black, thick] (C.south) 
      to[out=-90, in=180] ($(S1.south) + (0, -1)$) 
      -- (ExitPoint |- S1.south) 
      -- ++(1,0) node[right, black] {NEXT};
      
  \node[below right, font=\small] at (C.south east) {FALSE};

  \end{tikzpicture}
  \caption{Flusso di controllo del ciclo While.}
\end{figure}

\textbf{Spiegazione del diagramma:}
\begin{itemize}
    \item Il flusso entra nell'etichetta $L1$ (inizio del \texttt{while}).
    \item Viene valutata la condizione $C$.
    \item Se $C$ è \textbf{TRUE}, si va all'etichetta $L2$, si esegue lo statement $S_1$ e, al termine di $S_1$, il flusso deve tornare indietro a $L1$ per rivalutare la condizione.
    \item Se $C$ è \textbf{FALSE}, il flusso salta direttamente all'uscita ($S.next$ o NEXT).
\end{itemize}

\subsection*{Schema di Traduzione (SDT)}
Inserendo le azioni semantiche all'interno della produzione per supportare la generazione durante il parsing (anche top-down), otteniamo:

\begin{verbatim}
S -> while (
    { L1 = new(); L2 = new(); C.false = S.next; C.true = L2; }
    C )
    { S1.next = L1; }
    S1
    { S.code = label || L1 || C.code || label || L2 || S1.code; }
\end{verbatim}

In questo modo, le etichette $L1$ e $L2$ vengono generate prima di analizzare la condizione, e l'attributo $S_1.next$ viene impostato correttamente prima di analizzare il corpo $S_1$.

\subsection{Implementazione di SDD L-attribute}

I seguenti sono metodi che realizzano la traduzione durante la visita di un albero di parsing

\begin{itemize}
    \item \textbf{Traduzione durante la visita di un albero di parsing:}
    \begin{enumerate}
        \item Costruzione di un albero di parsing annotato (funziona per qualsiasi SDD non circolare).
        \item Costruzione dell'albero, aggiunta delle azioni e loro esecuzione in pre-ordine.
    \end{enumerate}
    \textit{Nota:} Questo approcci hanno un costo computazionale elevato (costruzione dell'albero).

    \item \textbf{Metodi di traduzione durante il parsing:}
    \begin{enumerate}
        \item Utilizzo di un parser a discesa ricorsiva con una funzione per ogni non-terminale.
        \item Generazione del codice al volo (On-the-fly), utilizzando un parser a discesa ricorsiva.
        \item Implementazione di uno SDT insieme a un parser LL.
        \item Implementazione di uno SDT insieme a un parser LR.
    \end{enumerate}
\end{itemize}

\subsubsection{Traduzione durante il parsing a discesa ricorsiva}
Un parser a discesa ricorsiva prevede una funzione A() per ogni non-terminale A della grammatica. È possibile estendere un tale parser
e trasformarlo in un traduttore facendo in modo che:
\begin{itemize}
    \item Gli argomenti della funzione $A()$ siano gli \textbf{attributi ereditati} del simbolo non-terminale $A$.
    \item Il valore restituito da $A()$ sia l'insieme degli \textbf{attributi sintetizzati} del simbolo $A$.
\end{itemize}

Il corpo della funzione deve occuparsi sia del parsing che della gestione degli attributi, in particolare la funzione deve:
\begin{enumerate}
    \item Decidere quale produzione utilizzare per espandere $A$.
    \item Verificare che ogni simbolo terminale appaia in ingresso quando richiesto; in quanto segue assumeremo che non sia necessario effettuare backtracking.
    \item Conservare in variabili locali i valori degli attributi necessari per il calcolo degli attributi ereditati relativi ai non-terminali nel corpo della produzione e/o
    degli attributi ereditati relativi al non-terminale alla testa della produzione.
    \item Chiamare le funzioni corrispondenti ai non-terminali nel corpo della produzione, passando gli argomenti corretti (ereditati) e salvando i risultati (sintetizzati).
\end{enumerate}

\subsubsection{Esempio: Implementazione \texttt{while} con ritorno di stringa}
\begin{lstlisting}[caption={Pseudocodice per la traduzione del while}]
  string S(label next) {
      string Scode, Ccode; /* variabili locali per frammenti di codice */
      label L1, L2;        /* etichette locali */
        
      if (il simbolo_corrente è il token WHILE) {
        
          avanza il puntatore di ingresso;
            
          verifica che '(' sia il prossimo simbolo, quindi avanza;
            
          L1 = new(); // Etichetta inizio controllo condizione
          L2 = new(); // Etichetta inizio corpo del ciclo
            
          Ccode = C(next, L2); 
            
          verifica che ')' sia il prossimo simbolo, quindi avanza;
            
          Scode = S(L1); 
            
          // Costruzione dell'attributo sintetizzato finale
          return ("label" || L1 || Ccode || "label" || L2 || Scode);
      }
      else { 
          /* gestione altri tipi di statement */ 
      }
  }
  \end{lstlisting}

  \subsection{Esempio Pratico: Parser a Discesa Ricorsiva per Espressioni}
  Di seguito è riportato il codice (pseudocodice C-like) per un parser che gestisce espressioni aritmetiche con eliminazione della ricorsione sinistra e gestione degli attributi.
  
  \textbf{Grammatica di riferimento (dopo eliminazione ricorsione sinistra):}
  \begin{align*}
  E &\to T \{ R.i = T.v \} R \{ E.v = R.s \} \\
  R &\to + T \{ R_1.i = R.i + T.v \} R_1 \{ R.s = R_1.s \} \mid \epsilon \{ R.s = R.i \} \\
  T &\to F \{ X.i = F.v \} X \{ T.v = X.s \} \\
  X &\to * F \{ X_1.i = X.i * F.v \} X_1 \{ X.s = X_1.s \} \mid \epsilon \{ X.s = X.i \} \\
  F &\to ( E ) \{ F.v = E.v \} \mid d \{ F.v = d.lexval \}
  \end{align*}
  
  \textbf{Implementazione delle funzioni:}
  
  \begin{lstlisting}
  // Gestione Produzione E -> T R
  int E() {
      int v_i, ev;
      // FIRST(T R) include FIRST(F) = { (, d }
      if (lookahead in { '(', 'd' }) {
          v_i = T();       // Calcola T.val
          ev = R(v_i);     // Passa T.val come ereditato a R
          return ev;       // Ritorna attributo sintetizzato
      } else {
          error();
      }
  }
  
  // Gestione Produzione T -> F X
  int T() {
      int x_i, tv;
      if (lookahead in { '(', 'd' }) {
          x_i = F();       // Calcola F.val
          tv = X(x_i);     // Passa F.val come ereditato a X
          return tv;
      } else {
          error();
      }
  }
  
  // Gestione Produzione R -> + T R1 | epsilon
  int R(int v_i) {
      int v_if, vs, tv;
      if (lookahead == '+') {
          match('+');
          tv = T();           // Calcola valore del termine successivo
          v_if = v_i + tv;    // Calcola nuovo ereditato: somma parziale
          vs = R(v_if);       // Ricorsione
          return vs;
      } 
      // Gestione produzione epsilon (FOLLOW(R) include ')', '$')
      else if (lookahead in { ')', '$' }) {
          return v_i;         // Ritorna il valore accumulato finora
      } else {
          error();
      }
  }
  
  // Gestione Produzione X -> * F X1 | epsilon
  int X(int x_i) {
      int x_1i, xs, fv;
      if (lookahead == '*') {
          match('*');
          fv = F();           // Calcola valore del fattore
          x_1i = x_i * fv;    // Calcola nuovo ereditato: prodotto parziale
          xs = X(x_1i);       // Ricorsione
          return xs;
      }
      // Gestione produzione epsilon (FOLLOW(X) include '+', ')', '$')
      else if (lookahead in { '+', ')', '$' }) {
          return x_i;
      } else {
          error();
      }
  }
  
  // Gestione Produzione F -> ( E ) | d
  int F() {
      int fv;
      if (lookahead == '(') {
          match('(');
          fv = E();
          match(')');
          return fv;
      } 
      else if (lookahead == 'd') {
          fv = lexval; // Valore del numero
          match('d');
          return fv;
      } else {
          error();
      }
  }
  \end{lstlisting}
  

  \subsubsection{Generazione del codice al volo}

  In molti casi comuni, tra cui il nostro esempio di generazione del codice del costrutto \texttt{while}, è possibile costruire incrementalmente porzioni di codice e memorizzarle in un array o in un file mediante opportune azioni dello SDT. Per fare ciò, le seguenti condizioni devono essere soddisfatte:
  
  \begin{enumerate}
      \item Per uno o più non-terminali esiste un attributo \textit{principale}. Per semplicità assumeremo che gli attributi principali siano tutti di tipo stringa.
      
      \item Gli attributi principali sono sintetizzati.
      
      \item Le regole per la valutazione degli attributi principali garantiscono che:
      \begin{enumerate}
          \item[(a)] l'attributo principale è dato dal concatenamento degli attributi principali dei non-terminali che appaiono nel corpo della produzione più, eventualmente, altri elementi che non sono attributi principali;
          
          \item[(b)] gli attributi principali dei non-terminali appaiono nella regola nello stesso ordine in cui i non-terminali appaiono nel corpo della produzione.
      \end{enumerate}
  \end{enumerate}
  
  Tali condizioni implicano che l'attributo principale può essere costruito emettendo solamente gli elementi del concatenamento che non sono attributi principali.

\subsubsection{Esempio: Statement While}
Possiamo modificare la funzione in modo che emetta gli elementi dell'attributo principale $S.code$ invece di salvarli per poi concatenarli nel valore di $S.code$ che verrà poi restituito.

Le funzioni $S()$ e $C()$ non restituiscono alcun valore, poiché tutti i loro attributi sintetizzati sono prodotti mediante stampa. Inoltre, la posizione delle istruzioni di stampa nella funzione è importante. L'ordine in cui i vari elementi vengono stampati è il seguente: per prima cosa la stringa ``label'' $L1$, quindi il codice relativo al non-terminale $C$, la stringa ``label'' $L2$, e infine il codice derivante dalla chiamata ricorsiva della funzione $S()$ .

\textbf{Codice con generazione al volo (pseudocodice):}

\begin{lstlisting}
void S(label next) {
    label L1, L2; /* etichette locali */
    if (simbolo_corrente è il token WHILE) {
        avanza il puntatore di ingresso;
        verifica che '(' sia il prossimo simbolo, quindi avanza;
        
        L1 = new(); 
        L2 = new();
        
        print("label", L1); // Stampa immediata
        
        // C gestisce i salti: true -> L2 (corpo), false -> next (uscita)
        C(next, L2); 
        
        verifica che '(' sia il prossimo simbolo, quindi avanza;
        
        print("label", L2); // Stampa etichetta corpo
        
        S(L1); // Chiamata ricorsiva per il corpo, torna a L1 dopo esec.
    }
    else { /* altri statement */ }
}
\end{lstlisting}

\subsection{Trasformazione SDD L-attribuita su grammatica LL a SDT azioni interne alle produzioni}
Supponiamo che una SDD L-attribuita sia basata su una grammatica LL e che sia stata convertita in uno SDT in cui le azioni si trovano all'interno delle produzioni. Possiamo effettuare la traduzione durante il parsing LL a patto di estendere lo stack del parser in modo da poter contenere \textbf{diversi tipi di record}.

Oltre ai record che rappresentano i terminali e i non-terminali della grammatica, lo stack del parser conterrà:
\begin{itemize}
    \item \textbf{action-record}: cioè record che contengono puntatori ad azioni che devono essere eseguite, e anche copie di attributi.
    \item \textbf{synthesize-record}: ovvero record destinati a salvare gli attributi sintetizzati dei non-terminali. Possono contenere azioni il cui scopo, in ordine, è copiare gli attributi sintetici più in basso nello stack in altri record.
\end{itemize}

Per gestire gli attributi sullo stack ci baseremo sui seguenti principi:

\begin{itemize}
    \item Gli attributi \textbf{ereditati} di un non-terminale $A$ sono memorizzati sullo stack, nel record che rappresenta il non-terminale. Il codice necessario per la valutazione di tali attributi è in genere rappresentato mediante un \textit{action-record} memorizzato sullo stack, immediatamente al di sopra del record che rappresenta $A$.
    
    \item Gli attributi \textbf{sintetizzati} relativi al non-terminale $A$ sono memorizzati in un \textit{synthesize-record} separato e posizionato sullo stack immediatamente al di sotto del record relativo ad $A$.
\end{itemize}




\section{Generazione del Codice Intermedio}

In questa fase si considerano \textbf{3 aspetti} fondamentali:

\begin{itemize}
    \item \textbf{Rappresentazione intermedia}, che si divide in:
    \begin{itemize}
        \item Alberi sintattici / DAG
        \item Codice a 3 indirizzi
    \end{itemize}
    \item \textbf{Analisi sintattica} (controlli di tipo).
    \item \textbf{Generazione del codice intermedio}.
\end{itemize}

\subsection*{Flusso di compilazione (Front-end vs Back-end)}
Il seguente schema illustra il flusso dei dati attraverso i componenti del compilatore:

\begin{center}
\begin{tikzpicture}[
    node distance=1cm and 1cm,
    auto,
    block/.style={
        rectangle, 
        draw, 
        text width=2cm, 
        align=center, 
        minimum height=3em
    },
    line/.style={draw, -latex}
]

    % Nodi
    \node (start) {};
    \node [block, right=0.5cm of start] (parser) {Parser};
    \node [block, right=of parser] (checker) {Checker\\statico};
    \node [block, right=of checker] (gen_int) {Generatore\\di codice\\intermedio};
    \node [block, right=2.5cm of gen_int] (gen_cod) {Generatore\\di codice};
    \node [right=0.5cm of gen_cod] (end) {};

    % Frecce
    \path [line] (start) -- (parser);
    \path [line] (parser) -- (checker);
    \path [line] (checker) -- (gen_int);
    \path [line] (gen_int) -- node [above, font=\footnotesize] {codice} node [below, font=\footnotesize] {intermedio} (gen_cod);
    \path [line] (gen_cod) -- (end);

    % Linee Front-end / Back-end
    % Linea orizzontale sotto
    \draw [thin] ($(parser.south west) + (0,-0.8)$) -- ($(gen_cod.south east) + (0,-0.8)$);
    
    % Separatore verticale
    \draw [thick] ($(gen_int.south east) + (1.25,-0.7)$) -- ($(gen_int.south east) + (1.25,-0.9)$);
    
    % Etichette
    \node [font=\footnotesize] at ($(checker.south) + (0.5,-1.1)$) {front-end};
    \node [font=\footnotesize] at ($(gen_cod.south) + (-0.5,-1.1)$) {back-end};

\end{tikzpicture}
\end{center}

\subsection{Analisi Statica (Static Checker)}
L'analisi statica è un insieme di controlli di consistenza effettuati al momento della compilazione per garantire che il programma possa essere compilato con successo e per individuare errori prima dell'esecuzione. Si divide in:

\subsubsection{1. Controlli Sintattici}
Verificano il rispetto delle regole strutturali del linguaggio. Esempi:
\begin{itemize}
    \item Un identificatore deve essere dichiarato al più una volta nello stesso scope.
    \item Un'istruzione \texttt{break} deve trovarsi all'interno di un ciclo (\texttt{while}, \texttt{for}) o \texttt{switch}.
    \item Distinzione tra \textbf{L-value} e \textbf{R-value}:
    \begin{itemize}
        \item \textbf{L-value} (Left-value): Indica una locazione di memoria (es. a sinistra di un assegnamento: \texttt{i = 5}).
        \item \textbf{R-value} (Right-value): Indica il valore contenuto nella locazione (es. a destra di un assegnamento: \texttt{x = i + 1}).
    \end{itemize}
\end{itemize}

\subsubsection{2. Controlli di Tipo}
Garantiscono che operatori e funzioni siano applicati a un numero corretto di operandi e che il loro tipo sia adeguato.
\begin{itemize}
    \item Esempio di conversione implicita (coercion):
    \begin{itemize}
        \item \texttt{x = 4 + 5} : Nessuna conversione necessaria.
        \item \texttt{x = 4 + 5.1} : Il numero intero \texttt{4} deve essere convertito in float prima della somma.
    \end{itemize}
\end{itemize}

\subsection{Rappresentazioni Intermedie e DAG}

La rappresentazione intermedia serve da ponte tra il front-end e il back-end. Esistono diversi livelli di astrazione:

\begin{center}
\begin{tikzpicture}[node distance=0.5cm, auto, font=\small]
    % Nodi del flusso
    \node (src) {Prog. Sorgente};
    \node[right=0.8cm of src] (high) {Rapp. Alto Livello};
    \node[right=0.8cm of high] (dots) {$\dots$};
    \node[right=0.8cm of dots] (low) {Rapp. Basso Livello};
    \node[right=0.8cm of low] (tgt) {Codice Target};

    % Frecce flusso
    \draw[->] (src) -- (high);
    \draw[->] (high) -- (dots);
    \draw[->] (dots) -- (low);
    \draw[->] (low) -- (tgt);

    % Annotazioni (Albero e 3 Indirizzi)
    \node[below=0.8cm of high, text=magenta] (tree_label) {Albero Sintattico};
    \draw[->, magenta, bend right] (tree_label.north) to (high.south);

    \node[below=0.8cm of low, text=magenta] (addr_label) {Codice a 3 indirizzi};
    \node[below=0.1cm of addr_label, font=\scriptsize] {$x = y \text{ op } z$};
    \draw[->, magenta, bend right] (addr_label.north) to (low.south);
\end{tikzpicture}
\end{center}

\begin{itemize}
    \item \textbf{Albero Sintattico:} Rappresentazione gerarchica fedele alla grammatica.
    \item \textbf{DAG (Grafo Aciclico Diretto):} Simile all'albero sintattico, ma identifica le sotto-espressioni comuni. I nodi con lo stesso valore/struttura vengono riutilizzati (parte condivisa).
\end{itemize}

\subsubsection*{Esempio di DAG}
Consideriamo l'espressione: $a + a * (b - c) + (b - c) * d$.
La sotto-espressione $(b - c)$ viene calcolata una sola volta e il nodo risultante viene puntato da entrambe le parti che lo utilizzano.

\begin{center}
\begin{tikzpicture}[
    level distance=1.5cm,
    sibling distance=1.5cm,
    every node/.style={circle, inner sep=1pt},
    edge from parent/.style={draw, thick}
]

    % Definizione manuale dei nodi per creare la condivisione
    \node (root) at (0,0) {$+$};
    
    \node (plus_left) at (-2, -1.5) {$+$};
    \node (mult_right) at (2, -1.5) {$*$};
    
    \node (a_first) at (-3, -3) {$a$};
    \node (mult_left) at (-1, -3) {$*$};
    \node (d) at (3, -3) {$d$};
    
    \node (a_second) at (-2, -4.5) {$a$};
    
    % Nodo condiviso (evidenziato in rosa come negli appunti)
    \node (minus_shared) at (0, -4.5) {$-$};
    \node (b) at (-0.5, -6) {$b$};
    \node (c) at (0.5, -6) {$c$};

    % Cerchio rosa attorno alla parte condivisa
    \draw[magenta, thick] (0, -5.2) ellipse (1cm and 1.5cm);
    \node[magenta, below=1.5cm of minus_shared] {$\uparrow$ parte condivisa};

    % Collegamenti
    \draw (root) -- (plus_left);
    \draw (root) -- (mult_right);
    
    \draw (plus_left) -- (a_first);
    \draw (plus_left) -- (mult_left);
    
    \draw (mult_right) -- (minus_shared); % Condivisione destra
    \draw (mult_right) -- (d);
    
    \draw (mult_left) -- (a_second);
    \draw (mult_left) -- (minus_shared); % Condivisione sinistra
    
    \draw (minus_shared) -- (b);
    \draw (minus_shared) -- (c);

\end{tikzpicture}
\end{center}


\end{document}